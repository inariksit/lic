\chapter*{Abstract}\label{chp:abstract}

%% Abstract page

%% The third page must have a short reference (abstract).

%% The reference must:
%% Be written in English.
%% Contain the name of the dissertation, possible subtitle, name of the author, department and 'Chalmers University of Technology'.
%% If the thesis is written in Swedish this should also be stated.
%% Be concise and do not exceed more than approximately 250 words.
%% Provide a brief, easily understood overview of the essential content of the thesis (problems, methods, results).
%% Conclude with a maximum of ten key words of significance for computerised information systems.

%% The reference can also be printed on the back of the presentation sheet, see below.

% \textbf{\lictitle}\\
% \textit{\licsubtitle}\\
% \textsc{\licauthor}\\
% \licdepartment\\
% \textsc{\licuniversity}\\

Constraint Grammar (CG) is a relatively young formalism, born out of practical 
need for a robust and language-independent method for part-of-speech
tagging.
This thesis presents two contributions to the field of CG.
We model CG as a Boolean satisfiability (SAT) problem, and
describe an implementation using a SAT solver.
This is attractive for several reasons: formal logic is well-studied,
and serves as an abstract language to reason about the properties of
CG. 

As a practical application, we use SAT-CG to analyse existing CG
grammars, taking inspiration from software verification. 

% \bigskip
% \noindent
% \textbf{Keywords}: \emph{\lickeywords}
