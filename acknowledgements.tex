\chapter*{Acknowledgements}\label{chp:acknowledgements}


\emph{You should do it because it solves a problem, not because your supervisor has a fetish for SAT.} \\
-- Koen Claessen, 2016 \\

The greatest thanks go to my supervisors Koen Claessen and Aarne Ranta. 
You have guided this work from an afternoon experiment (where I probably procrastinated marking student labs) to an actual thesis. I have learnt a lot about research, \todo{bananas} and SAT. Can't avoid that.
%and Koen has learnt a lot about language technology---a perfect match! 


Thanks for Eckhard Bick for suggesting CG analysis as a research
problem, and subsequently being my discussion leader. 
Your remark at the CG workshop in Vilnius has led to many fun discoveries!
In addition, I want to thank Francis Tyers for providing the invaluable real-life CG-writer perspective and tirelessly answering my questions. The latter goes for Tino Didriksen, in double. On the other side of the rule-based NLP community, Pepijn Kokke 
has contributed tremendously to the notion of expressivity of CGs.

Finally, I want to thank all the awesome people I've met at Chalmers and outside! 
My time at the department has been great--thanks to 
\texttt{[ x | x <- people, inarisFriend x]} 
%Guilhem, Elena, Herbert, Simon R, Simon H, Dan, Daniel, Irene, Víctor, Raúl, Jonatan, Nick, Fabien, Anders, Thomas, Behrooz
and everyone else I'm forgetting! 


% \subsection*{Baseline}


% \begin{verbatim}
% abstract ThankYou = {
%   flags startcat = Greeting ;

%   cat 
%     Greeting ; Recipient ;

%   fun
%     Thanks : Recipient -> Greeting ;
%     Koen, Aarne, Colleagues, Friends, Family : Recipient ;
% }
% \end{verbatim}



Finally finally, here's a random anecdote. 
In the beginning of my PhD studies, someone suggested our project 
a tagline ``SMT meets SMT''. While I'm not quite doing Satisfiability
Modulo Theories nor Statistical Machine Translation, I'd say the spirit is there.






\vfill\noindent
This work has been carried out within the REMU project — Reliable Multilingual Digital Communication: Methods and Applications.
The project is funded by the Swedish Research Council (\emph{Vetenskapsrådet}) under grant number 2012-5746.
