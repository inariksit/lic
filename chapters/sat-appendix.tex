\appendix
\chapter{SAT-encoding}

\section{SAT-encoding of sentences}

We receive input from the morphological analyser in the following format.

\begin{verbatim}
"<wordform1>"
      "lemma1" morpho1 morpho2
      "lemma2" morpho3 morpho4
"<wordform2>"
      "lemma3" morpho5 morpho6
      "lemma4" morpho7 morpho8
\end{verbatim}

We transform each reading into a SAT-variable:

\begin{centering}
$wordform1/lemma1<morpho1><morpho2>$
$wordform1/lemma2<morpho3><morpho4>$
$wordform2/lemma3<morpho5><morpho6>$
$wordform2/lemma4<morpho7><morpho8>$
\end{centering}

Then, for each cohort, we require that at least one of its readings is true. We add the following clauses:

\begin{centering}
$wordform1/lemma1<morpho1><morpho2> \vee wordform1/lemma2<morpho3><morpho4>$
$wordform2/lemma3<morpho5><morpho6> \vee wordform2/lemma4<morpho7><morpho8>$
\end{centering}



\section{SAT-encoding of rules}

\subsubsection{Unordered scheme}

\subsubsection{Ordered scheme}
