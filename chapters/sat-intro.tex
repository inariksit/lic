\section{Boolean satisfiability (SAT)}


Imagine you are in a pet shop with the following selection of animals.

{\center Ant, bee, cat, dog, elephant, f..., gorilla}

These animals are very picky about each others' company. The dog has no teeth and needs the cat to chew its food. The cat, in turn, wants its best friend f.... 
But the bee is very aggressive towards all carnivores, and the elephant is afraid of anything lighter than 10 kilograms. The gorilla hates four-legged creatures, and the cat doesn't trust anything that stings. 

You need to decide on a subset of pets to buy---you love all animals, but realistically, you cannot have them all. You start calculating in your head: ``If I take the gorilla, I cannot take cat, dog, nor elephant. How about I take the dog, I can take everything but the gorilla and the bee.''
After some time, you decide on ant, cat, dog and f..., leaving the bee, the elephant and the gorilla in the pet shop.


This everyday situation is an example of a \emph{SAT-problem}.
The animals translate into \emph{variables}: $\{ant, bee, cat, dog, elephant, f..., gorilla\}$.
The lists of friends and enemies of each animal translate into \emph{clauses}.
These clauses are disjunctions of \emph{literals}: either a variable ($cat$) or its negation ($\neg{}cat$).
The literal $cat$ means that the animal comes with you, 
and $\neg{}cat$ means it is left in the shop. 
The clause $\neg{}dog \vee cat$ means ``I don't buy the dog or I buy the cat''. 
Some examples of the clauses follow:

\begin{center}
$\neg{}elephant \vee ant \vee cat \vee bee$ \\
$\neg{}dog \vee \neg{}gorilla$ \\
$\neg{}dog \vee \neg{}bee$ \\
\end{center}

The objective is to find a \emph{model}: each variable is assigned a Boolean value, such that all clauses evaluate into true. We can see that the assignment 

{\center $\{ant=True, bee=False, cat=True, dog=True, elephant=False, f...=True, gorilla=False\}$ }
\noindent satisfies the animals' wishes.
\todo{This model is the only solution to the animal problem}. However, some problems have multiple solutions, and some are unsatisfiable, i.e. they have no solution at all.


\paragraph{What it is and what it can do}


\begin{quote}
Satisfiability solving, the problem of deciding whether the variables of a propositional formula can be assigned in such a way that the formula evaluates to true, is one of the classic problems in computer science. It is of theoretical interest because it is the canonical NP-complete problem. It is of practical interest because modern SAT-solvers can be used to solve many important and practical problems.\end{quote} \\
-- Claessen et al, 200?



\paragraph{Applications of SAT}

Model checking

Since those works, SAT solving techniques have improved significantly \cite{marques_silva2010}, and they are used in domains such as microprocessor design and computational 
biology---these problems easily match or exceed CG in complexity. 
Thanks to these advances, we were able to revisit the idea and develop it
further. 


\paragraph{What does it help you to formulate something as a SAT-problem?}

SAT-solving is all about searching.
SAT-problems are in general NP-complete. However, a typical real-life scenario formulated as SAT-problem can be solved much more efficiently with proper heuristics \cite{claessen_al}.
Formulating a search problem in SAT gives us access to decades of research into SAT-solving. 
Instead of implementing our own search, we can exploit decades of research that has been devoted to SAT-solving: optimising the search and finding the best heuristics, as well as innovative ways of formulating SAT-problems.