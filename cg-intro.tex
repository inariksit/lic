
Constraint Grammar (CG) is a tool for disambiguating text which has various
possible analyses for each word. It was first introduced by
\cite{karlsson1995constraint}, and has been used for many tasks in
computational linguistics, such as POS tagging, surface syntax and
machine translation \cite{bick2011}. It disambiguates output by
morphological analyser by using
constraint rules which can select or remove a potential analysis
(called `reading') for a
target word, depending on the context words around it. Together these rules disambiguate the whole text.

In the example below, I show possible analyses for the sentence ``the bear
sleeps'':


\begin{verbatim}
"<the>"
        "the" det def
"<bear>"
        "bear" noun sg
        "bear" verb pres
        "bear" verb inf
"<sleeps>"
        "sleep" noun pl
        "sleep" verb pres p3 sg
\end{verbatim}

\noindent We can disambiguate this sentence with two rules:

\begin{enumerate}
\def\labelenumi{\arabic{enumi}.}
\itemsep1pt\parskip0pt\parsep0pt
\item \texttt{REMOVE verb IF (-1 det)}
  `Remove verb after determiner'
\item  \texttt{REMOVE noun IF (-1 noun)}
  `Remove noun after noun'
\end{enumerate}

\noindent Rule 1 matches the word \emph{bear}: it is tagged as verb and is
preceded by a determiner. The rule removes both verb readings from
\emph{bear}, leaving it with an unambiguous analysis \texttt{noun sg}.
Rule 2 is applied to the word \emph{sleeps}, and it removes the noun
reading. The finished analysis is shown below:


\begin{verbatim}
"<the>"
        "the" det def
"<bear>"
        "bear" noun sg
"<sleeps>"
        "sleep" verb pres p3 sg
\end{verbatim}

It is also possible to use syntactic tags and (mostly rudimentary) phrase
structure in the analyses. In that
case, after disambiguating \emph{bear} by part of speech (noun), it
remains to disambiguate whether it is a subject, object, adverbial or
any of the possible syntactic roles. An example disambiguation could
look like the following:


\begin{verbatim}
"<the>"
        "the" det def  \textbf{<BEGIN-NP @Det}
"<bear>"
        "bear" noun sg  \textbf{@Subj END-NP>}
"<sleeps>"
        "sleep" verb pres p3 sg  \textbf{@Pred}

\end{verbatim}

With the introduction of CG-3, it is also possible to add dependency
relations: number the words in the sentence and for each set a parent,
such as UNCOMMENT THIS LATER%\texttt{"<the>" "the" det def  \textbf{IND=1 PARENT=2}}. 
However, these features are recent and not used in many of the grammars written 
in the community. In this introduction, I will illustrate my examples with the 
most basic operations, that is, disambiguating morphological tags.


\section*{Properties of Constraint Grammar}\label{properties}

All theories and implementations of CG agree that is a \emph{reductionist}
system with \emph{shallow} syntax. There are different takes on how
\emph{deterministic} the rules are: the state-of-the-art CG parser VISL CG-3
executes the rules strictly based on the order they appear, but there
are other implementations that apply their own heuristics, or remove the
ordering in total, such as in finite-state or logic-based implementations. 
In the following, I will address the three features and relate CG to
other formalisms, in terms of the feature in question.

\subsection*{Reductionist vs.~licencing}\label{reductionist-vs.licencing}

CG is a reductionist system: it starts from a set of alternative
analyses, and eliminates the impossible or improbable ones using
constraint rules. The remaining analyses are assumed to be correct; that
is, everything that is not explicitly eliminated, is allowed. This can
be contrasted with a licencing system, where constructions must
be explicitly allowed, otherwise they are illegal. An empty
reductionist grammar will accept any string, whereas an empty
licencing grammar will accept no string.


\subsection*{Shallow vs.~deep structure}\label{shallow-vs.deep-structure}

CG is shallow for various reasons.

\paragraph{Locality of context} The rules do not usually handle 
long-distance dependencies; they operate on a unit of a few words, and
don't abstract away from surface features such as word order.
More importantly, CG doesn't introduce a full tree structure to the
whole sentence. Some local subtrees can be detected, but mostly
for the purposes of aiding the disambiguation.
 
\paragraph{Little enforcement of grammaticality} CG doesn't
per se define sequences as \emph{grammatically correct}.
Beside the lack of long-distance dependencies, there is
no mandatory enforcement of even local well-formedness, such as gender agreement.
However, this is often helpful for practical purposes, because it adds robustness.
Let us illustrate with an example in Swedish.

\begin{verbatim}
"<en>"
        "en" det indef utr sg
        "en" noun utr sg 
"<bord>"
        "bord" noun neutr sg
\end{verbatim}

The first word, \emph{en}, is ambiguous between the indefinite determiner for
\emph{utrum} gender  (masculine and feminine collapsed into one), or a noun (`juniper').
The second word, \emph{bord}, is a neuter noun (`table')---the writer has
most likely meant to write ``a table'' instead of ``juniper table'', but has mistaken the gender of `table'. The correct form, which also would not be ambiguous, is ``ett bord''.

CG allows the rules to be as fine-grained or broad as we want: 
\texttt{SELECT det IF (1 noun) (0 noun)} would match any determiner and any noun, 
and succesfully disambiguate \emph{en} in this case. 
We can also enforce grammaticality by writing \texttt{SELECT det utr sg IF (1 noun utr sg)}%\footnote{the counterpart neuter determiner is not ambiguous with noun, so \texttt{SELECT det neutr sg IF (1 noun neutr sg) (0 noun)} would be redundant. Though it can be ambiguous with the numeral `1'. So actually maybe it's good anyway.}
. In that case, nothing in the example phrase matches, and it is left ambiguous.


%On the other hand, we can introduce such fine-graind rules in order to pick the right analysis, for instance, if a certain modifier is underspecified. If we analyse Spanish text \emph{la casa grande}, where \emph{grande} is underspecified for gender, then having a rule that picks the feminine adjective after a feminine article and feminine noun makes our analysis of \emph{grande} more accurate. 
% a certain adjective is underspecified for gender but its head noun is specified, then we can pick the right gender for the adjective.

The previous example illustrates that the notions of \emph{accept} and \emph{reject} are not clear: if agreement was enforced
by enumerating only legal combinations, the grammar would just refuse to disambiguate
an ungrammatical sequence and leave all tags intact---in that case,
its performance would not differ from a grammar that simply does not have many rules.
If we define \emph{grammatical sequences} to be sequences of POS tags,
then the question is somewhat easier to answer: \texttt{\textless{}det\textgreater{}\textless{}utr\textgreater{} \textless{}noun\textgreater{}\textless{}neutr\textgreater{}} is ungrammatical, and CG doesn't raise an alarm that there is something wrong. However, the alternative \texttt{\textless{}noun\textgreater{}\textless{}utr\textgreater{} \textless{}noun\textgreater{}\textless{}neutr\textgreater{}} might be in that particular context even ``more wrong''; same goes for the option with no disambiguation at all.
Bottom line: it's not about defining grammaticality, it's providing analyses in any case.

% Of course, the latter point can be made of any reductionist system, not only CG.

\paragraph{Some relevant heading for this part} CG has a more lightweight task than e.g.~a phrase structure grammar: a
successful disambiguation can depend on just a local context, in a
window of 1-3 words left or right. To demonstrate the the difference,
let us take the same the example sentence \emph{the bear sleeps}, and a
context-free grammar with the following productions:


\begin{verbatim}
S   -> NP VP
VP  -> V NP
NP  -> Det N
Det -> "the"
N   -> "bear" | "sleeps"
V   -> "bear" | "sleeps"
\end{verbatim}

There is no way to reach \texttt{S} if \emph{bear} is analysed as
\texttt{V} and \emph{sleeps} as \texttt{N}; thus we get the individual
words disambiguated thanks to the parser, which introduces a structure
to the whole sentence. If one of the words was out of vocabulary, say
\emph{the bear warbles}, there would be no analysis for \emph{the} or
\emph{bear} either. In CG, an unrecognised verb would be no problem at
all in disambiguating the previous words.


Contrasting with phrase-structure or dependency grammars, the rules in
CG are often on a lower level of abstraction. 
The rule that tells to remove verb reading after a determiner
doesn't tell us about the structure of noun phrase; we need a second
rule to remove a verb after a determiner and an adjective (``the happy
bear sleeps''), and a third rule to remove verb after a possessive
pronoun (``my bear sleeps''). A phenomenon that is expressed with one
big rule in a deep formalism such as HPSG or categorial grammar, is
split into many low-level rules.

On the other hand, these features can provide
flexibility that is hard to mimic by a deeper formalism.
For instance, rules can target individual words
% (select \texttt{noun} if the word form is \emph{bear})
or other properties that are not generalisable to a whole word class,
such as verbs that express cognitive processes.
Introducing a subset of verbs, even if they are used only in one rule,
is very cheap and doesn't create a complicated inheritance hierarchy
of different verb types.
We can introduce a similar granularity in other systems, for instance,
a grammar in Grammatical Framework, which would include the functions
\texttt{GenVP} for whichever verbs and \texttt{CogVP} for cognitive verbs:


\begin{verbatim}
GenVP : V2    -> NP -> VP ;
CogVP : CogV2 -> NP -> VP ;
\end{verbatim}

In order to parse as much text as without the subcategorisation, the
grammar writer would have to add \texttt{CogV*} counterparts all
functions that exist for \texttt{V*}, which duplicates code; or add
coercion functions from  \texttt{CogV*} to \texttt{V*}, which increases ambiguity.
In case of multiple parses, likely the one with more information
would be preferred, but the CG solution gives more flexibility. If a
rule which matches \texttt{CogV} is introduced before (more on
ordering in the following section) a rule which matches \texttt{V}, it
is applied before, and vice versa. This allows for very fine control,
for example combining the semantic properties of the verb and the
animacy of the arguments.

\paragraph{Theoretical principles after all these practical examples} This lack of deep structure is intentional in the design of CG.
\cite{karlsson1995constraint} justifies the choice with a number of
theoretical principles: that language is open-ended and grammars are
bound to leak, and that necessary information for the syntactic analysis
comes from the morphology, hence morphology is ``the cornerstone of
syntax''. In a CG, one has access to all levels at the same time,
ranging from morphology to syntax or dependency labels, or even
semantics, if one wants to introduce semantic roles in the
tagset. One can have a rule such as \texttt{REMOVE ... IF (+1 "bear" <noun>
  @Object §Patient)}, followed by a second rule which only considers the POS of the
context word.
%  From an engineering point of view, even an incomplete CG can
% already disambiguate a lot, whereas an incomplete phrase structure
% grammar is nearly useless.

\subsection*{Ordering of the rules}\label{ordering}

The previous properties of Constraint Grammar formalism and rules were specified in \cite{karlsson1995constraint}. However, Karlsson omits some practical details regarding implementation, one being the order in which the rules are executed. After the initial specification, there have been several independent implementations of CG, with different ordering schemes. In the following section, we will present the different parameters of ordering and test their performance, using a previously written grammar.

The execution of the rules depends on the following parameters: strict vs. heuristic ordering, and sequential vs. parallel execution. These parameters can combine freely: strict and sequential, strict and parallel, heuristic and sequential, or heuristic and parallel. Throughout the section, we will apply the rules to the following ambiguous passage:
\begin{verbatim}
"<what>"
        "what" determiner
        "what" pronoun                                        
 "<question>"
        "question" noun
        "question" verb
 "<is>"
        "be" verb
 "<this>"
       "this" determiner
\end{verbatim}

\paragraph{Strict vs. heuristic}

This aspect concerns the ordering of the rules in the file. An implementation with strict order applies each rule in the order in which they appear in the file. If a grammar contains the rules “select noun after determiner” and “select verb after pronoun” in the given order, the rule that selects noun will be applied first. After it has selected the noun reading for question, there are no verb readings available anymore for the second rule to fire.
					
How do we know which rule is the right one? There can be many rules that fit the context, but we choose the one that just happens to appear first in the rule file. More careful rules are usually placed first, followed by stronger rules—see the pattern below.

\begin{verbatim}					
SELECT <rare analysis> IF <rare condition> ;
...
REMOVE <rare analysis> ;
\end{verbatim}
If a rare condition is met, the rare analysis is selected, and since it will be the only analysis, it will be protected from the remove rule. If the rare condition is not met, the rare analysis is removed.
					
An alternative solution to a strict order is to use heuristic order: when disambiguating a particular word, find the rule that has the longest and most detailed match. Now, assume that there is a rule with a longer context, such as “select noun if -1 is determiner and +1 is verb”, even if this rule appears last in the file, it would be preferred to the shorter rules, because it is a more exact match.
					
Both methods have their strengths and weaknesses. A strict order is more predictable, but it also means that the grammar writers need to pay more thought to rule interaction. A heuristic order frees the grammar writer from finding an optimal order, but it can give unexpected results, which are harder to debug.


\paragraph{Sequential vs. parallel}

The input sentence can be processed in sequential or parallel manner. In sequential execution, the rules are applied to one word at a time, starting from the beginning of the sentence. The sentence is updated after each application. If the word \emph{what} gets successfully disambiguated as a pronoun, then the word \emph{question} will not match the rule ``select noun after determine''.

In contrast, a parallel execution disambiguates all the words at the same time, using their initial, ambiguous context. Both ``select noun after determiner'' and ``select verb after pronoun'' will be applied to question, because the original input from the morphological analyser contains both determiner and pronoun as the preceding word.

Which execution scheme is better? For most purposes, a sequential execution is preferred.  Sequential execution–--together with strict or heuristic rule order---makes the rule take action immediately, and updates the context for the following words. If we know the correct part-of-speech for \emph{what}, then it is easier to disambiguate \emph{question}.
Parallel execution may have uses in more marginal cases: for instance, if we have a small rule set and we want to find out if the rules are consistent with each other. 



% The constraint rules are executed in sequence, and the consequences of
% previous rule applications affect the coming rules. In addition, there
% is a default rule, which overrides all other rules: the last reading of
% a word must not be removed.
% 
% If a grammar contains the rules ``select \texttt{noun} after
% \texttt{det}'' and ``select \texttt{verb} after \texttt{det}'',
% whichever occurs first will be applied; after it has selected the noun
% reading and removed the rest, there is no verb reading available anymore
% for the second rule to fire.
% 
% More careful rules are usually placed first, followed by stronger
% rules---see the pattern below. If a rare condition is met, the rare
% analysis is selected, and since it will be the only analysis, it will be
% protected from the remove rule. If the rare condition is not met, the
% rare analysis is removed.
% 
% 
% \begin{verbatim}
% SELECT <rare analysis> IF <rare condition> ;
% ...
% REMOVE <rare analysis> ;
% \end{verbatim}
% 
\paragraph{Comparison to FSIG} 
Another example of a reductionist and shallow-syntax formalism is
Finite-State Intersection Grammar (FSIG) \citep{koskenniemi90}. In the
formalism, the tagged sequences of words, as well as rules, are modelled
as finite state automata: for instance, there are three transitions from
the state \emph{bear} to the state \emph{sleeps} in the automaton
representing \emph{the bear sleeps}. This automaton is intersected with
a rule automaton, and each path through the resulting automaton
represents a possible analysis for the whole sequence.
 
A parallel and unordered rule set in CG corresponds to FSIG.
\cite{koskenniemi90} explicitly says that an FSIG grammar must be
consistent:
 
 \begin{quote}
 Each constraint simply adds something to the discriminating power of the
 whole grammar. No constraint rule may ever forbid something that would
 later on be accepted as an exception. This, maybe, puts more strain for
 the grammar writer but gives us better hope of understanding the grammar
 we write.
 \end{quote}
 
% \cite{lager_nivre01} illustrate the difference between CG and FSIG with
% the following example. Suppose we have the same rule that removes \texttt{verb}
% after \texttt{det}, and we apply it to the sequence \emph{what
% question}:
% 
% 
% \begin{verbatim}
% "<what>"
        % det
        % pron
% "<question>"
        % noun
        % verb
% \end{verbatim}
% 
% 
% In FSIG, the path that consists of choosing the \texttt{det} arc
% followed by \texttt{verb} arc is ruled out, but other interpretations
% are still valid: \texttt{det+noun}, \texttt{pron+noun} and
% \texttt{pron+verb}. In contrast, applying a rule in CG removes an
% analysis permanently; if the rule removes the verb reading in
% \emph{question}, then we cannot retrieve the \texttt{pron+verb}
% possibility. Therefore, it is important to consider the order of the
% rules. It is possible to operate with \emph{cautious context}, where the
% rule only takes action if the context is unambiguous. For this case, the
% ``remove \texttt{verb} after \texttt{det}'' rule would not fire before
% the word \emph{what} is fully disambiguated as a determiner.
% 

\section*{Encoding in logic}\label{encoding-in-logic}

\cite{lager98} represents a CG-like, shallow and reductionist system in
logic. \cite{lager_nivre01} builds on that in a study which reconstructs
four formalisms in logic. CG is contrasted with Finite-State
Intersection Grammar (FSIG) and Brill tagging; all three work on a set
of constraint rules which modify the initially ambiguous input, but with some crucial
differences.

The rules and analyses are represented as clauses, and if it is
possible to build a model which satisfies all clauses, then there is
an analysis for the sentence.
Take the unordered case first. The authors define predicates \emph{word}
and \emph{\sc{pos}}, and form clauses as follows. The variable $P$ is used to
denote any word, and the index $n$ its position.

\begin{align*}
\text{word(P, the)} \  &\Rightarrow \ \text{{\sc pos}(P, det)} \\
\text{word(P, bear)} \ &\Rightarrow \ \text{{\sc pos}(P, verb)} \vee \text{{\sc pos}(P, noun)} \\
\text{{\sc pos}(P$_n$, det)} \wedge \text{{\sc pos}(P$_{n+1}$, verb)} \ &\Rightarrow 
\end{align*}

\noindent The first clauses read as ``if the word is \emph{the}, it
is a determiner'' and ``if the word is \emph{bear}, it can be a verb or a noun''.
The third clause represents the rule which prohibits a verb after a
determiner. It normalises to $\neg \text{{\sc pos}(P$_n$, det)} \vee \neg \text{{\sc pos}(P$_{n+1}$, verb)}$, and we know that {\sc pos}(P, det) must be true for
the word \emph{the}, thus the verb analysis for \emph{bear} must be
false.

This representation models FSIG, where the rules are logically
unordered. For CG, the authors introduce a new predicate for each rule,
$\text{{\sc pos}}^i$, where $i$ indicates the index of the rule in the
sequence of all rules.
Each rule is translated into two
clauses: 1) the conditions hold, the target has >1 analysis\footnote{In \cite{listenmaa_claessen2015}, the default rules
  are given to the SAT solver as separate clauses; for every word
  $w$, at least one of its analyses \{$w_1$,$w_2$,...,$w_n$\} must be
  true. Then the clauses for each rule don't need to repeat the ``only
  if it doesn't remove the last reading'' condition.}, and the 
targetet reading is selected or removed; and
2) the conditions don't hold or the target has only one analysis, and the target is left untouched. The general form of the clauses is shown below:

\begin{align*}
\text{{\sc pos}}^i\text{(P, T)} \: \wedge \: (\;\:\,  \text{conditions\_hold}
\wedge  \text{|T|} > 1) \  &\Rightarrow \  \text{{\sc pos}}^{i+1}\text{(P, T$\,\setminus\,$[target])} \\
\text{{\sc pos}}^i\text{(P, T)} \: \wedge \: (\neg \text{conditions\_hold} \vee
\text{|T|} = 1) \  &\Rightarrow \ \text{{\sc pos}}^{i+1}\text{(P, T)}
\end{align*}


To show a concrete example, the following shows the two rules in the specified order:
\begin{inparaenum}[\itshape a\upshape)]
\def\labelenumi{\arabic{enumi}.}
\itemsep1pt\parskip0pt\parsep0pt
\item \texttt{REMOVE verb IF (-1 det)}; and
\item  \texttt{REMOVE noun IF (-1 noun)}
\end{inparaenum}.

\begin{align*}
\text{{\sc pos}$^1$(P, [det])} \  &\Leftarrow \  \text{word(P, the)} \\
\text{{\sc pos}$^1$(P, [verb,noun])} \ &\Leftarrow \ \text{word(P, bear)} \\
\text{{\sc pos}$^1$(P, [verb,noun])} \ &\Leftarrow \ \text{word(P, sleeps)} \\ \\
\text{{\sc pos}$^2$(P$_n$, T$\,\setminus\,$[verb])} \ &\Leftarrow \ \text{{\sc pos}$^1$(P$_n$, T)}
 \wedge (\;\; \text{{\sc pos}$^1$(P$_{n-1}$, [det])}  \wedge  \text{T$\,\setminus\,$[verb] $\neq$ []}) \\
\text{{\sc pos}$^2$(P$_n$, T)} \ &\Leftarrow \  \text{{\sc pos}$^1$(P$_n$, T)} \wedge
(\neg \text{{\sc pos}$^1$(P$_{n-1}$, [det])} \vee \text{T$\,\setminus\,$[verb]
  $=$ []}) \\ \\
\text{{\sc pos}$^3$(P$_n$, T$\,\setminus\,$[noun])} \ &\Leftarrow \ \text{{\sc pos}$^2$(P$_n$, T)}
 \wedge (\;\; \text{{\sc pos}$^2$(P$_{n-1}$, [noun])}  \wedge  \text{T$\,\setminus\,$[noun] $\neq$ []}) \\
\text{{\sc pos}$^3$(P$_n$, T)} \ &\Leftarrow \  \text{{\sc pos}$^2$(P$_n$, T)} \wedge
(\neg \text{{\sc pos}$^2$(P$_{n-1}$, [noun])} \vee \text{T$\,\setminus\,$[noun]
  $=$ []})
\end{align*}

The logical reconstruction helps to provide some clarity when comparing CG
to FSIG. In FSIG, the predicate ${\sc pos}$ is a statement of an analysis of a
word; in case of uncertainty, disjunction is used to present all possible analyses. In CG, uncertainty is modelled by sets of analyses,
and the predicate ${\sc pos}^i$ is a statement of the
set of analyses of a word at a given stage of the rule sequence. The
final result is obtained by composition of these clauses in the order of
the rule sequence.


%The implementation in \cite{listenmaa_claessen2015} is not quite CG, not quite FSIG.

\section*{Conclusion}

In the paper I've presented an introduction to Constraint Grammar,
highlighting some key properties and contrasting it to other grammar
formalisms.
CG is relatively young, popularised 20 years ago, and throughout its
history, it has been in favour of linguists more than computer
scientists, which might explain why its formal background has not been
studied extensively---notable exceptions being \cite{lager98} and \cite{lager_nivre01}.
 
Is CG a formalism? In order to be a grammar formalism, a candidate
framework should discriminate between grammatical and ungrammatical
sequences, and provide a structure to the grammatical sequences.
Let us start from the latter criterion. Clearly a successful run of a
high-quality rule set leaves us with more structure than we started
with. In the beginning, we know that \emph{bear} could be a verb
(infinitive or present tense, any number or person except 3rd
singular) or a noun (subject, object, predicative, ...), and in the end, we
will know the right analysis for it, given that the rule set is
good. The requirement of a good rule set doesn't affect CG's status as
a formalism or not; one could as well write a terrible context-free
grammar that doesn't correspond to natural language at all.
If POS tags are not enough to consider as a structure,
CG can also be used to manipulate syntactic analyses and even to add
dependency relations, using exactly the same mechanisms (``set parent to +1 if the word at 0 is \emph{bear} and -1 is a determiner'').

The first criterion can be harder to justify. \cite{bick2015} describe CG as ``a
declarative whole of contextual possibilities and impossibilities for
a language or genre'', which is nevertheless implemented in a
low-level way: selecting and removing readings from individual words,
without explicit connection between the rules. Bick and Didriksen
argue that as a side effect, the rules actually manage to describe language.
CG seems to even have some interest as a tool for psycholinguistics research:
there is a Russian being developed within the research group
 \href{https://uit.no/forskning/forskningsgrupper/gruppe?p_document_id=344365}{CLEAR
} (Cognitive Linguistics: Empirical Approaches to Russian); accessing
all levels from morphology to semantics at the same time is particularly attractive to cognitive linguists (personal communication).
As for its descriptive power, we can again compare CG to CFG. If
badly constructed, both CG and CFG can analyse almost everything,
making it useless for language description. If we accept that a
possibility to construct a bad grammar doesn't revoke CFG's status as
a formalism, it shouldn't do so for CG either. One can always compare
the relative ease of constructing a good grammar between formalisms,
but that is another topic already.

As a bottom line, CG provides a lightweight framework where already a
moderate amount of rules can be useful (e.g. \cite{lene_trond2011} with 115
rules), although high quality requires in general thousands of rules.
Like any formalism, CG itself leaves the users with much freedom; a
single grammar can be anything from completely ungrammatical rules to near-perfect description of language.

% For example \cite{bick2015} mentions a ``mature morphosyntactic core
% grammar'' which has 6000 rules. Assuming this is a typical number of
% rules, one could compare the efforts of writing those rules to writing
% a grammar of similar coverage and precision in another formalis


% This lack of structure is intentional in the design of CG;
% \cite{karlsson1995constraint} cites a number of theoretical principles,
% including

% \begin{quote}
% Language is an open-ended system where there is no strict demarcation
% line between grammatical and ungrammatical sentences. Therefore all
% grammars are bound to leak.
% \end{quote}

% \begin{quote}
% The cornerstone of syntax is morphology {[}--{]}. Syntactic rules are
% generalisations telling a) how word-forms occur in particular word order
% configurations, and b) what natural classes, ''syntactic functions'',
% can be isolated and inferred in such configurations.
% \end{quote}

% The first point also justifies a reductionist approach in favour of
% licencing; the second highlights that information comes from the
% morphological analyses