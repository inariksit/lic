


Our work is primarily inspired by \cite{lager98, lager_nivre01}, both encoding CG in logic.
\cite{lager98} presents a ``CG-like, shallow and reductionist system'' translated into a disjunctive logic program.
\cite{lager_nivre01} builds on that in a study which reconstructs
four different formalisms in first-order logic. 
CG is contrasted with Finite-State Intersection Grammar (FSIG) \cite{koskenniemi90} 
and Brill tagging \todo{cite}; all three work on a set of constraint rules 
which modify the initially ambiguous input, but with some crucial differences.
%
Other works combining logic to CG include
\cite{lindberg_eineborg98ilp} and \cite{asfrent14}, both using Inductive Logic Programming to learn CG rules from a tagged corpus.
In addition, \cite{lager01transformation} \todo{something about ordering of the CG rules.}



In the following, we describe the encoding used by 

\todo{I think this is the most important part to save, if something? \\
The logical reconstruction in \cite{lager_nivre01} helps to provide some clarity when comparing CG to FSIG.
In FSIG, the predicate ${\sc pos}$ is a statement of an analysis of a
word; in case of uncertainty, disjunction is used to present all possible analyses. In CG, uncertainty is modelled by sets of analyses,
and the predicate ${\sc pos}^i$ is a statement of the
set of analyses of a word at a given stage of the rule sequence. The
final result is obtained by composition of these clauses in the order of
the rule sequence.}

%%%%%%%%%%%%%%%%%%%%%%

% The rules and analyses are represented as clauses, and if it is
% possible to build a model which satisfies all clauses, then there is
% an analysis for the sentence.
% Take the unordered case first. The authors define predicates \emph{word}
% and \emph{\sc{pos}}, and form clauses as follows. The variable $P$ is used to
% denote any word, and the index $n$ its position.

% \begin{align*}
% \text{word(P, the)} \  &\Rightarrow \ \text{{\sc pos}(P, det)} \\
% \text{word(P, bear)} \ &\Rightarrow \ \text{{\sc pos}(P, verb)} \vee \text{{\sc pos}(P, noun)} \\
% \text{{\sc pos}(P$_n$, det)} \wedge \text{{\sc pos}(P$_{n+1}$, verb)} \ &\Rightarrow 
% \end{align*}

% \noindent The first clauses read as ``if the word is \emph{the}, it
% is a determiner'' and ``if the word is \emph{bear}, it can be a verb or a noun''.
% The third clause represents the rule which prohibits a verb after a
% determiner. It normalises to $\neg \text{{\sc pos}(P$_n$, det)} \vee \neg \text{{\sc pos}(P$_{n+1}$, verb)}$, and we know that {\sc pos}(P, det) must be true for
% the word \emph{the}, thus the verb analysis for \emph{bear} must be
% false.

% This representation models FSIG, where the rules are logically
% unordered. For CG, the authors introduce a new predicate for each rule,
% $\text{{\sc pos}}^i$, where $i$ indicates the index of the rule in the
% sequence of all rules.
% Each rule is translated into two
% clauses: 1) the conditions hold, the target has >1 analysis\footnote{In \cite{listenmaa_claessen2015}, the default rules
%   are given to the SAT solver as separate clauses; for every word
%   $w$, at least one of its analyses \{$w_1$,$w_2$,...,$w_n$\} must be
%   true. Then the clauses for each rule don't need to repeat the ``only
%   if it doesn't remove the last reading'' condition.}, and the 
% targetet reading is selected or removed; and
% 2) the conditions don't hold or the target has only one analysis, and the target is left untouched. The general form of the clauses is shown below:

% \begin{align*}
% \text{{\sc pos}}^i\text{(P, T)} \: \wedge \: (\;\:\,  \text{conditions\_hold}
% \wedge  \text{|T|} > 1) \  &\Rightarrow \  \text{{\sc pos}}^{i+1}\text{(P, T$\,\setminus\,$[target])} \\
% \text{{\sc pos}}^i\text{(P, T)} \: \wedge \: (\neg \text{conditions\_hold} \vee
% \text{|T|} = 1) \  &\Rightarrow \ \text{{\sc pos}}^{i+1}\text{(P, T)}
% \end{align*}


% To show a concrete example, the following shows the two rules in the specified order:
% \begin{enumerate}
% \item[] \begin{verbatim}REMOVE verb IF (-1 det) ;
% REMOVE noun IF (-1 noun) ;
% \end{verbatim}
% \end{enumerate}

% \begin{align*}
% \text{{\sc pos}$^1$(P, [det])} \  &\Leftarrow \  \text{word(P, the)} \\
% \text{{\sc pos}$^1$(P, [verb,noun])} \ &\Leftarrow \ \text{word(P, bear)} \\
% \text{{\sc pos}$^1$(P, [verb,noun])} \ &\Leftarrow \ \text{word(P, sleeps)} \\ \\
% \text{{\sc pos}$^2$(P$_n$, T$\,\setminus\,$[verb])} \ &\Leftarrow \ \text{{\sc pos}$^1$(P$_n$, T)}
%  \wedge (\;\; \text{{\sc pos}$^1$(P$_{n-1}$, [det])}  \wedge  \text{T$\,\setminus\,$[verb] $\neq$ []}) \\
% \text{{\sc pos}$^2$(P$_n$, T)} \ &\Leftarrow \  \text{{\sc pos}$^1$(P$_n$, T)} \wedge
% (\neg \text{{\sc pos}$^1$(P$_{n-1}$, [det])} \vee \text{T$\,\setminus\,$[verb]
%   $=$ []}) \\ \\
% \text{{\sc pos}$^3$(P$_n$, T$\,\setminus\,$[noun])} \ &\Leftarrow \ \text{{\sc pos}$^2$(P$_n$, T)}
%  \wedge (\;\; \text{{\sc pos}$^2$(P$_{n-1}$, [noun])}  \wedge  \text{T$\,\setminus\,$[noun] $\neq$ []}) \\
% \text{{\sc pos}$^3$(P$_n$, T)} \ &\Leftarrow \  \text{{\sc pos}$^2$(P$_n$, T)} \wedge
% (\neg \text{{\sc pos}$^2$(P$_{n-1}$, [noun])} \vee \text{T$\,\setminus\,$[noun]
%   $=$ []})
% \end{align*}

% The logical reconstruction helps to provide some clarity when comparing CG
% to FSIG. In FSIG, the predicate ${\sc pos}$ is a statement of an analysis of a
% word; in case of uncertainty, disjunction is used to present all possible analyses. In CG, uncertainty is modelled by sets of analyses,
% and the predicate ${\sc pos}^i$ is a statement of the
% set of analyses of a word at a given stage of the rule sequence. The
% final result is obtained by composition of these clauses in the order of
% the rule sequence.