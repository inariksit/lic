\def\la{\text{\em la}}
\def\casa{\text{\em casa}}
\def\grande{\text{\em grande}}

\def\det{{\text{\sc Det}}}
\def\prn{{\text{\sc  Prn}}}
\def\n{{\text{\sc N}}}
\def\v{{\text{\sc V}}}
\def\adj{{\text{\sc Adj}}}

\def\laDet{\la_\det}
\def\laPrn{\la_\prn}
\def\casaN{\casa_\n}
\def\casaV{\casa_\v}
\def\grandeAdj{\grande_\adj}

\def\t#1{\texttt{#1}}

\def\defRule{``do not remove the last reading''}


\chapter{CG as a SAT-problem}
\label{chapterCGSAT}

In this chapter, we present CG as a Boolean satisfiability (SAT) problem,
and describe an implementation using a SAT-solver. 
This is attractive for several reasons: formal logic is
well-studied, and serves as an abstract language to reason about the
properties of CG. Constraint rules encoded in logic capture richer
dependencies between the tags than standard CG. 
%; much like in parallel CG, one rule is not a self-contained unit, but a piece in the puzzle.


So far, we have presented CG and SAT as separate success stories: CG is easy to adopt, even for less-resourced languages, and achieves high F-scores; SAT is used in reducing difficult search problems into a low-level task.
But is there a reason to combine the two? 

%CG lends itself well to a logical representation. It's all about expressing what is true (`SELECT') and false (`REMOVE'), under certain conditions (`IF').
Applying logic to reductionist grammars has been explored earlier by \cite{lager98,lager_nivre01}, but there has not been, to our knowledge, a full logic-based CG implementation; at the time, logic programming was too slow to be used for tagging or parsing. 
Since those works, SAT-solving techniques have improved significantly \cite{marques_silva2010}, and they are used in domains such as microprocessor design and computational 
biology---these problems easily match or exceed CG in complexity. 
In addition, SAT-solving brings us more practical tools

Thanks to these advances, we were able to revisit the idea and develop it further. 


The content in this chapter is based on \cite{listenmaa_claessen2015}; 
however, the ordered scheme described here is different from the one in \cite{listenmaa_claessen2015}.
We include evaluation for three schemes: the original unordered, the original ordered, and the updated ordered.
Other changes to the content are just added detail.



\section{Previous work: CG in logic}\label{encoding-in-logic}

\input{chapters/cg-sat-relatedwork}


\section{CG as a SAT-problem}
\label{sec:CGSAT}

First we present the preliminaries: Input=ambiguous clauses; Each reading = variable; Apply rule = create implications on those variables; Goal: find a solution!

Then, we explain different ordering schemes, and discuss conflict handling.

Finally, we evaluate SAT-CG against VISL CG-3, for both F-score and running time.
%: for F-score, we use a 20,000--word Spanish gold standard corpus, and for running time, we use.

\subsection{Preliminaries}

%We have seen in Section~\ref{sec:SAT-intro} how to transform a simple decision task into a SAT-problem. 
%:each animal was represented as a variable, and their restrictions for cohabiting were expressed as clauses; ``if dog, then cat''.
In the following, we translate the disambiguation of a sentence into a SAT-problem.
We demonstrate our encoding with an example in Spanish, shown in Figure~\ref{fig:laCasaGrande}: {\em la casa grande}. % (`the big house'). 
The first word, {\em la}, is ambiguous between a definite article (`the') or an object pronoun (`her'), and the second word, {\em casa}, can be a noun (`house') or a verb form (`marries').
The subsegment {\em la casa} alone can be either a noun phrase, $\laDet \ \casaN$ 
`the house'  or a verb phrase $\laPrn \ \casaV$   `(he/she) marries her'. 
However, the unambiguous adjective, {\em grande} (`big'), disambiguates the whole segment into a noun phrase: `the big house'.

\begin{figure}[h]
\centering
\begin{tabular}{p{0.6cm} l | c | c }
%\multicolumn{2}{c}{}
   & \textbf{Original~analysis} 
                & \textbf{Variables}
                              & \textbf{Default rule} \\ \hline
\t{"<la>"}   &   &            &  {\small \defRule} \\
  & \t{"el" 
  det def f sg}  & $\laDet$   &  \\
  & \t{"lo" 
  prn p3 f sg}   & $\laPrn$   &   $\laDet \vee \laPrn$ \\
\t{"<casa>"} &   &            &   \\
  & \t{"casa" 
  n f sg}        & $\casaN$   &  \\
  & \t{"casar"
   v pri p3 sg}  & $\casaV$   & $\casaN \vee \casaV$  \\
\t{"<grande>"} & &            & \\
  & \t{"grande" 
  adj mf sg}   & $\grandeAdj$ & $\grandeAdj$
\end{tabular}
\caption{Ambiguous segment in Spanish: translation into SAT-variables.}
\label{fig:laCasaGrande}
\end{figure}


% \begin{figure}[h]
% \centering
% \begin{verbatim}
% "<la>"
%         "el" det def f sg
%         "lo" prn p3 f sg
% "<casa>"
%         "casa" n f sg
%         "casar" v pri p3 sg
% "<grande>"
%         "grande" adj mf sg
% \end{verbatim}
% \caption{Ambiguous segment in Spanish.}
% \label{fig:laCasaGrande}
% \end{figure}

\paragraph{Reading}
The readings of the word forms make a natural basis for variables.
We translate a combination of a word form and a reading, such as \texttt{"<la>" ["el" det def f sg]}, into a variable $\laDet$, which represents the possibility that \la{} is a determiner. This example segment gives us five variables: $\{ \laDet , \laPrn , \casaN , \casaV,  \grandeAdj \}$, shown in \ref{fig:laCasaGrande}.

\paragraph{Cohort} As in the original input, the readings are grouped together in cohorts. We need to keep this distinction, for instance, to model {\sc select} rules and cautious context: 
\t{SELECT "casa" n} means, in effect, ``remove $\casaV$'', and \t{IF (-1C prn)} means ``if $\laPrn$ is true and $\laDet$ false''. 
%
Most importantly, we need to make sure that the last reading is not removed. Hence we add the default rule, \defRule, as shown in the third column of \ref{fig:laCasaGrande}. 
These disjunctions ensure that at least one variable in each cohort must be true.



\paragraph{Sentence}
In order to match conditions against analyses, the input needs to be structured as a sentence: the cohorts must follow each other like in the original input, indexed by their absolute position in the sentence. Thus when we apply \texttt{REMOVE v IF (-1 det)} to the cohort $2 \rightarrow [\casaN , \casaV]$, the condition will match on $\laDet$ in cohort 1.


\paragraph{Rule}

Next, we formulate a rule in SAT. A single rule, such as \texttt{REMOVE v IF (-1 det)}, is a template for forming an implication; when given a concrete sentence, it will pick concrete variables by the following algorithm.

\begin{enumerate}
\item Try rule to all cohorts
 \begin{itemize}
    \item[\la:] No target found
    \item[\casa:] Target found in $\casaV$, try to match conditions
      \begin{itemize}
       \item Condition found in previous cohort, $\laDet$
       \item Create a clause: $\laDet \Rightarrow \neg \casaV \ $ `if \la{} is a determiner, \casa{} is not a verb'
      \end{itemize}
    \item[\grande:] No target found
  \end{itemize}
\item Solve: given the clauses by default rule, and clauses formed by rule application.
\end{enumerate}

\paragraph{Disambiguation}

Finally, we want to see the model given by the SAT-solver: the disambiguated sentence. We started off with all the variables in place, but did not actually give them values: just required that at least one variable in each cohort must be true.
In addition, we gave the clause $\laDet \Rightarrow \neg \casaV$. We can see with bare eye that this problem will have a solution; in fact, multiple ones. 
The verb analysis is removed in the first two columns, as required by the presence of $\laDet$. However, the rule may as well be interpreted ``if $\casaV$ may not follow $\laDet$, better remove $\laDet$ instead''; this has happened in columns 3--4. 
We see a third interpretation in column 5: $\laDet \Rightarrow \neg \casaV$ is only an 
implication, not an equivalence; this means that $\casaV$ may be removed even without 
the presence of $\laDet$. 

\begin{figure}[h]
\centering
$$\begin{array}{ c | c | c | c | c}
 1 & 2 & 3 & 4 & 5 \\ \hline
 \laDet   &  \laDet  &         &        &        \\
          &  \laPrn  & \laPrn  & \laPrn & \laPrn \\
 \casaN   &  \casaN  & \casaN  &        & \casaN \\
          &          & \casaV  & \casaV &         \\
\grandeAdj & \grandeAdj & \grandeAdj & \grandeAdj & \grandeAdj \\

\end{array}$$
\caption{Ambiguous segment in Spanish: all possible models.}
\label{fig:laCasaGrandeSolved}
\end{figure}

The reader may notice that this is not a particularly exciting result. A standard CG would just do what it is told and remove the verb, not give 5 different interpretations for a single rule.

But there is power to this property. Now, imagine we add a second rule: \texttt{REMOVE n IF (-1 prn)}. \todo{all the magic that happens in the NODALIDA paper \cite{listenmaa_claessen2015}. Also tell that this is exactly what FSIG does, and what \cite{lager98} does. And that we will do more!}

We have given a minimal description of SAT-based implementation. 
Many details are left vague: Do we enforce that all readings that are not targeted by rules will resolve to true? How do we treat ordering? 





\subsection{Unordered scheme}

\subsection{Ordered scheme}


% We add the following rules:

% \begin{itemize}
% \item [] \texttt{REMOVE prn IF (1 n) ;} \\
%              \texttt{REMOVE det IF (1 v) ;}
% \end{itemize}

% Standard CG will apply one  of the rules to the word \emph{la}; 
% either the one that comes first, or by some other heuristic. 
% The other rule will not fire, because it would remove the last
% reading. 
% If we use the cautious mode (\texttt{1C n} or \texttt{1C v}), which
%  requires the word in the context to be fully disambiguated, 
% neither of the rules will be applied.
% In any case, all readings of \emph{casa} are left untouched by these rules.

% The SAT solver performs a search, 
% and starts building possible models that satisfy both constraints. 
% In addition to the given constraints, we have default rules to emulate
% the CG principles: an analysis is true if no rule affects it,
% and at least one analysis for each word is true---the notion of
% ``last'' is not applicable.

% With these constraints, we get two solutions. The interaction of the
% rules regarding \emph{la}  disambiguates the part of speech of
% \emph{casa} for free, and the order of the rules does not matter. 

% \begin{enumerate}
% \item [\texttt{1)}]
% \begin{verbatim}
% "<la>"
%         "el" det def f sg
% "<casa>"
%         "casa" n f sg
% \end{verbatim}
% \item [\texttt{2)}]
% \begin{verbatim}
% "<la>"
%         "lo" prn p3 f sg
% "<casa>"
%         "casar" v pri p3 sg
%         "casar" v imp p2 sg
% \end{verbatim}
% \end{enumerate} 

% Without more context or additional rules we cannot fully disambiguate the passage,
% but unlike with standard CG, we get only legal combinations in one solution.



\noindent The most important differences between the traditional and
the SAT-based approach are described in the following sections.

\subsection{Rules disambiguate more}
Considering our example phrase and rules, the standard CG implementation
can only remove readings from the target word (\texttt{prn} or
\texttt{det}). The SAT-based implementation interprets the rules as
``determiner and verb together are illegal'', and is free to take action that concerns also the word in the condition (\texttt{n} or \texttt{v}).

This behaviour is explained by simple properties of logical formulae.
When the rules are applied to the text, they are translated into
implications: \texttt{REMOVE prn IF (1 n)} becomes
 \emph{casa}\texttt{<n>} $\Rightarrow$ $\neg$\emph{la}\texttt{<prn>},
 which reads ``if the \texttt{n} reading for \emph{casa} is true, then
 discard the \texttt{prn} reading for \emph{la}''.
Any implication $a\,\Rightarrow\,b$ can be represented as a disjunction
$\neg a\,\vee\,b$; intuitively, either the antecedent is false
and the consequent can be anything, or the consequent is true and the
antecedent can be anything.
Due to this property, our rule translates into the disjunction 
$\neg$\emph{casa}\texttt{<n>} $\vee$ $\neg$\emph{la}\texttt{<prn>},
which is also equivalent to another implication, 
 \emph{la}\texttt{<prn>} $\Rightarrow$ $\neg$\emph{casa}\texttt{<n>}.
This means that the rules are logically flipped: \texttt{REMOVE prn IF
  (1 n)} translates into the same logical formula as  \texttt{REMOVE n
  IF (-1 prn)}. 
A rule with more conditions corresponds to many rules, each condition
taking its turn to be the target. % of removal or selection.

% \begin{itemize}
% \item [] \texttt{REMOVE n IF (-1 prn) ;} \\
%          \texttt{REMOVE v IF (-1 det) ;}
% \end{itemize}

% SAT-based approach gives identical results with both sets of rules,
% whereas the standard CG would remove one reading from \emph{casa} and leave \emph{la} ambiguous.
% Testing this property with more complex rules and larger rule sets remains to be done.

%Rules with more conditions translate into many rules; rules with negation become complements (\texttt{(*)-X} for \texttt{NOT X}). Rules which require the condition to be unambiguously tagged, don't have an equivalent flip in the standard CG.
% We did not expect much practical benefits, save for realising that some rules are bad by having to think further what it implies,
%Out of interest, we manually flipped a small grammar and tested it against the original.
% \todo{The results show  }
%\begin{itemize}
%\item []          \texttt{SL Inf IF (-2 V-MOD) (-1 "not");} \\
%$\rightarrow$\texttt{SL V-MOD   IF (1 "not") (2 INF);} \\
%$\rightarrow$\texttt{SL "not" IF (-1 V-MOD) (1 Inf);}
%\item []           \texttt{SL Foo IF (NOT 1 Bar)} \\
%$\rightarrow$\texttt{SL (*)-Foo if (-1 Bar)}
%\end{itemize}


\subsection{Cautious context is irrelevant}
% Rather than waiting for a word to get disambiguated, the SAT solver starts by 
% making assumptions (e.g. ``\emph{casa} is a noun'') and working under them,
% discarding the assumption if it doesn't lead to a model that satisfies
% all constraints.

Traditional CG applies the rule set iteratively:
some rules fire during the first iteration, either because their
conditions do not require cautious context, or because some words are
unambiguous to start with. This makes some more words unambiguous, and
new rules can fire during the second iteration.

In SAT-CG, the notion of cautious context is irrelevant. Instead of
removing readings immediately,  each rule generates a  number of
implications, and the SAT solver tries to find a model that will satisfy them. 


Let us continue with the earlier example. 
We can add a word to the input:
\begin{itemize}
\item [] \emph{la casa grande} `the big house'
\end{itemize}
and a rule that removes verb reading, if the word is followed by an adjective:
\begin{itemize}
\item [] \texttt{REMOVE v IF (1 adj) ;}
\end{itemize}

% The SAT solver has already the clauses produced by the previous rules.
% \emph{casa}\texttt{<n>} $\Rightarrow$ $\neg$\emph{la}\texttt{<prn>}
% and   \emph{casa}\texttt{<v>} $\Rightarrow$ $\neg$\emph{la}\texttt{<det>}
The new rule adds the implication
 \emph{grande}\texttt{<adj>} $\Rightarrow$
 $\neg$\emph{casa}\texttt{<v>}, which will
disambiguate  \emph{casa} to a noun\footnote{Assuming that
  \texttt{adj} is the only reading for \emph{grande}, it must be true,
  because of the restriction that at least one analysis for each word
  is true. Then the implication has a true antecedent (\emph{grande}\texttt{<adj>}), thus its
  consequent ($\neg$\emph{casa}\texttt{<v>}) will hold.}. 
As the status of \emph{casa} is resolved, the SAT solver can now
discard the model where \emph{casa} is a verb and \emph{la} is a
pronoun
%---all these decisions are connected---
and we get a unique solution with \texttt{det n adj}.
% Now \emph{casa}\texttt{<n>} is true, so \emph{la}\texttt{<prn>} must be false. 
%Because \emph{la}\texttt{<prn>} is false,  \emph{casa} cannot
% be a \texttt{v}, and we get a unique solution with \texttt{det n adj}.

Contrast this with the behaviour of the standard CG.
With the new rule, standard CG will also remove the verb reading from \emph{casa}, 
but it is in no way connected to the choice for \emph{la}. It all
depends on the order of the two rules; if the \texttt{det} reading of
\emph{la} is removed first, then we are stuck with that choice.
If we made the first rules cautious, that is, keeping the determiner
open until \emph{casa} is disambiguated, then we get the same
result as with the SAT solver.
Ideally, both ways of grammar writing should yield similar results;
traditional CG rules are more imperative, and SAT-CG
rules are more declarative.

% The SAT-based approach only removes readings after it has enough
% evidence to do that. From the grammar writer perspective, this removes
% a burden of having to decide whether the rule should be cautious or
% not---the SAT solver will only take action the surrounding context
% supports the decision.

% \todo{Unordered rules: also not applicable anymore to apply rules
% iteratively; there's no ``this rule doesn't fire now but will after
% applying X and Y'', it's all just implications ``this rule will fire
% if this is true'' and let the SAT solver find if those rules can apply
% peacefully to the same input.}


\subsection{Rules can be unordered}
\label{sec:unord}

As hinted by the previous property, the SAT solver does not need a fixed
order of the rules.
Applying a rule to a sentence produces a number of clauses,
and those clauses are fed into the SAT solver.
However, in the unordered scheme, some information is lost: the
following rule sets would be treated identically, whereas in the
traditional CG, only the first would be considered as a bad order.

\begin{itemize}
\item [\texttt{1)}] \texttt{SELECT v ;} \\
         \texttt{REMOVE v IF (-1 det) ;} 
\item [\texttt{2)}] \texttt{REMOVE v IF (-1 det) ;} \\
         \texttt{SELECT v ;}
\end{itemize}

Without order, both of these rule sets will conflict, if applied to an
input that has sequence \texttt{det v}.
The SAT solver is given clauses that tell to select a verb and remove a
verb, and it cannot build a model that satisfies all of those clauses.
To solve this problem, we create a variable for every instance of rule application, and request a solution where maximally many of these variables are true.
If there is no conflict, then the maximal solution is one where all of
these variables are true; that is, all rules take action.


In case of a conflict, the SAT solver makes it possible to discard only
minimal amount of rule applications. Continuing with the example,
it is not clear which instances would be discarded, but if the rules
were part of a larger rule set, and in the context the \textsc{remove} rule was
the right one to choose, it is likely that the interaction between the desired
rules would make a large set of clauses that fit together, and the
\textsc{select} rule would not fit in, hence it would be discarded. 

This corresponds loosely to the common design pattern in
CGs, where there is a number of rules with the same target, ordered
such that more secure rules come first,
with a catch-all rule with no condition as the last resort, to be
applied if none of the previous has fired.
The order-based heuristic in the traditional CG is replaced by a more
holistic behaviour: if the rules conflict, discard the one that seems
like an outlier.


We can also emulate order with SAT-CG. To do that, we enter clauses
produced by each rule one by one, and assume the solver state reached
so far is correct. If a new clause introduces a conflict with
previous clauses, we discard it and move on to the next rule.
By testing against gold standard, we see that this scheme works better
with ready-made CGs, which are written with ordering in mind.
It also runs slightly faster than the unordered version.
\\



\noindent These three features influence the way rules are written. 
We predict that less rules are needed; whether this holds in the order
of thousands of rules remains to be tested. On the one hand, getting rid
of ordering and cautious context could ease the task of the grammar
writer, since it removes the burden of estimating the best sequence of
rules and whether to make them cautious. On the other
hand, lack of order can make the rules less transparent, and might not scale up for larger grammars.


\section{Evaluation}
\label{sec:eval}

For evaluation, we measure the performance against the state-of-the-art CG parser VISL CG-3.
% both in terms of accuracy and execution time.
SAT-CG fares slightly worse for accuracy, and significantly worse for execution time.
The results are presented in more detail in the following sections.
We include evaluation for three schemes: the original unordered, the original ordered, and the updated ordered.

%This suggests that our implementation should not compete with existing
%state-of-the-art, but rather it has value as a way of relating the CG
%formalism to wider context in the theory of computer science.
%In Section~\ref{sec:apps} we discuss more about possible applications.



\subsection{Performance against VISL CG-3}


We took a manually tagged
corpus\footnote{\url{https://svn.code.sf.net/p/apertium/svn/branches/apertium-swpost/apertium-en-es/es-tagger-data/es.tagged}}
containing approximately 22,000 words of Spanish news text, 
and a small constraint grammar\footnote{\url{https://svn.code.sf.net/p/apertium/svn/languages/apertium-spa/apertium-spa.spa.rlx}}, produced independently of the authors.
% We ignored substitute rules, but kept subreadings and unification, even though our implementation doesn't handle them.
We kept only \textsc{select} and \textsc{remove} rules, which left us 261 rules.
With this setup, we produced an ambiguous version of the tagged
corpus, and ran both SAT-CG and VISL CG-3 on it.
Treating the original corpus as the gold standard, the disambiguation
by  VISL CG-3 achieves F-score of 82.6 \%, ordered SAT-CG 81.5 \%  and
unordered SAT-CG 79.2 \%. 
We did not test with other languages or text genres due to the lack of
available gold standard.
%This result suggests that that when running grammars that are written
%with the traditional CG in mind, SAT-CG loses with both ordering
%strategies, but emulating order fares better.

We also tested whether SAT-CG outperforms traditional CG with a
small rule set. With our best performing and most concise
grammar\footnote{\url{https://github.com/inariksit/cgsat/blob/master/data/spa\_smallset.rlx}}
of only 19 rules, both SAT-CG and VISL CG-3  achieve a F-score of
around 85 \%. This experiment is very small and might be explained by
overfitting or mere chance, but it seems to indicate that rules that
work well with SAT-CG are also good for traditional CG.
% It was possible to attain a rule set for which the unordered SAT-CG
% works better (\textless0.5 \%), but it required trying
% subsequences of the 19-rule set in a brute force manner---this hardly
% presents any real life use case.



% Similar patterns were observed with small (\textless{}20) rule sets
% written by the authors; depending on the subset, SAT-CG and VISL CG-3 had
% a difference of at most $\Mypm$ 1.5 \%. 
% Introducing rules one by one up to 19, the
% performance improved in a very similar rate, with less than 0.5 \%
% difference between the systems at each new rule.
% We did not evaluate on other languages or text genres, due to lack of suitable test data.

% Additional tests could include plugging SAT-CG into Apertium
% translation pipeline, and comparing the translation quality.

\subsection{Execution time}

The worst-case complexity of SAT is exponential, whereas the standard
implementations of CG are polynomial, but with advances in SAT solving
techniques, the performance in the average case in practice is more feasible than in the previous works done in 90s--00s.
We used the open-source SAT solver MiniSat \cite{een04sat}.


We tested the performance by parsing Don Quijote (384,155 words) with
the same Spanish grammars as in the previous experiment. 
Table~\ref{table:time} shows execution times compared to VISL CG-3;
SAT-CG\textsubscript{u} is the unordered scheme and
SAT-CG\textsubscript{o} is the ordered.
% For both systems, the number of rules in the grammar affects the performance more than the word count.
From the SAT solving side, maximisation is the most costly operation. 
Emulating order is slightly faster, likely because the maximisation problems are smaller.
In any case, SAT does not seem to be the bottleneck: with 261 rules,
the maximisation function was called 147,253 times, and with 19 rules,
132,255 times, but 
%However, with both rule sets, half of the sentences in Don Quijote
%needed less than 6 calls of maximise, and 75 \% of the sentences
%needed less than 14.
the differences in the execution times are much larger, which suggests
that there are other reasons for the worse performance. 
This is to be expected, as SAT-CG is currently just a naive
proof-of-concept implementation with no optimisations.

\begin{table}
  \centering
  \begin{tabular}{|c|c|c|c|}
     \hline
     \textbf{\# rules} &  \textbf{SAT-CG\textsubscript{u}} & \textbf{SAT-CG\textsubscript{o}} & \textbf{VISL CG-3} \\ \hline
      19   & 39.7s &  22.1s   & 4.2s\\ %\hline
      99   & 1m34.1s & 1m14.9s & 6.1s \\ %\hline
      261  & 2m54.1s & 2m31.6s & 10.7s \\ \hline
  \end{tabular}
  \caption{Execution times for 384,155 words.}
  \label{table:time}
\end{table}

\section{Applications and future work}
\label{sec:apps}
%As demonstrated in Section~\ref{sec:eval}, our implementation is not
%competitive with the state of the art. 
%VISL CG-3 is efficient, reliable and widely used, and
Instead of trying to compete with the state of the art, we plan to use SAT-CG for grammar analysis\footnote{We thank Eckhard Bick for the idea.}.
There has been work on automatic tuning of hand-written CGs
\cite{bick2013tuning}, but to our knowledge no tools to
search for inconsistencies or suboptimal design. %in order to point it out to the grammar writer.

The sequential application of traditional CG rules is good for
performance and transparency. When a rule takes action, the analyses
are removed from the sentence, and the next rules get the modified
sentence as input. 
% At the execution of rule, there is no way to go back to earlier rules and undo them. 
As a downside, there is no way to know which part comes directly from the
raw input and which part from applying previous rules.

A conflict in an ordered scheme can be defined as 
a set of two or more rules, such that applying the first makes the
next rules impossible to apply, regardless of the input.
We can reuse the example from Section~\ref{sec:unord}:

\begin{itemize}
\item [] \texttt{SELECT v ;} \\
         \texttt{REMOVE v IF (-1 det) ;}
\end{itemize}

The first rule selects the verb reading everywhere and removes
all other readings, leaving no chance for the second rule to take action.
If the rules are introduced in a different order, there is no
conflict: the \textsc{remove} rule would not remove verb readings from all
possible verb analyses, so there is a possibility for the \textsc{select} rule
to fire.


Ordered SAT-CG can be used to detect these conflicts without any
modifications, as a side effect of its design. 
After applying each rule, it stores the clauses
produced by the rule and commits to them.
In case of a conflict, the program detects the particular rule that 
violates the previous clauses, with the sentence
where it is applied. Thus we get feedback which rule fails, and on which
particular word(s).

% This conflict solving can be done per section, or for the whole grammar. 
% In case of section-based application, we run each section on the
% input, and give the output as a fresh input to the new section, discarding
% all clauses. Alternatively, we can let the rules interact with each
% other regardless of the section boundaries.

Unordered SAT-CG with maximisation-based conflict solving is not
suitable for this task: the whole definition of conflict depends on
ordering, and the unordered scheme deliberately loses this information.
On a more speculative note, an unordered formalism such as Finite-State
Intersection Grammar \cite{koskenniemi90} might benefit from the
maximisation-based technique in conflict handling.

Finally, we intend to test for conflicts
without using a corpus.
Let us illustrate the idea with the same two rules,
\texttt{SELECT v} and \texttt{REMOVE v IF (-1 det)}
in both orders.
Assume we have the tag set \texttt{\{det, n, v\}}, 
and we want to find if there exists an input such that
 both rules, applied in the given order, remove something from the input.
% This is another constraint solving problem, hence we can model it in SAT.
There are no inputs that satisfy the requirement with the first order,
but several that work with the second, such as the following:

\begin{itemize}
\item []
\begin{verbatim}
"<w1>"
        det
        v
"<w2>"
        n
        v
\end{verbatim}
\end{itemize}
Thus we can say that the first rule order is conflicting, but the second
one is not. 
Implementing and testing this on a larger scale is left for future work.


% SELECT (a) IF (0 PALJON + Nom + Sg) (1 N + Nom + Sg) ; # PALJO VALITTAMINEN
% SELECT (a) IF (0 PALJON + Par) (1 N + Par + Sg) ;
% SELECT (a) IF (0 PALJON + LOC-CASE) (1 N + 1 LOC-CASE + Sg) ;
% REMOVE (a) IF (0 PALJON) ; # "on asetettu paljon haltijaksi"

\section{Conclusions}

SAT-solvers are nowadays powerful enough to be used for dealing with
Constraint Grammar. A logic-based approach to CG has possible
advantages over more traditional approaches; a SAT solver may
disambiguate more words, and may do so more precisely, capturing
more dependencies between tags.
We experimented with both ordered and unordered rules, and found the 
ordered scheme to work better with previously written grammars.
For future direction, we intend to concentrate on grammar analysis,
especially finding conflicts in constraint grammars.


% Also, the SAT-solver requires less rules, and these rules are simpler. 
% The ordering of rules is something we found was incompatible with a logic-based approach. We compensate this by maximising rule applications. Whether this is an advantage or disadvantage remains to be seen. Our initial experimental results are promising.

