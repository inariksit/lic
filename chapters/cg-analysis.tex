\chapter{CG analysis using SAT}

\paragraph{This is just random ramblings from my github.}

\subsection{Task}\label{task}

We create an initial symbolic sentence \texttt{w} that would make the
``last'' rule fire. Then we want to make \texttt{w} so that none of the
earlier rules has effect on it, because

\begin{enumerate}
\def\labelenumi{\arabic{enumi}.}
\itemsep1pt\parskip0pt\parsep0pt
\item
  conditions are out of scope (trivial, no clauses)
\item
  conditions in scope, but one or more doesn't hold
\item
  tag has been removed in target
\item
  all readings of target have the desired tag (cannot remove)
\end{enumerate}

\subsection{Examples}\label{examples}

Last rule:

\begin{verbatim}
   REMOVE inf  IF (-1 (prn pers))
\end{verbatim}

Earlier rules:

\begin{verbatim}
   REMOVE:r_pr_v pr  IF (1 vblex vbmod vaux vbhaver vbser ) (NOT 0 "te" "om te" )
   REMOVE:r_adv_n adv  IF (1 n np )
   SELECT:s_pr_v pr  + "te" "om te"  IF (1 vblex vbmod vaux vbhaver vbser  + inf )
\end{verbatim}

Examined rule creates a model with many solutions, among which the
following:

\begin{verbatim}
  "<w1>"
        prn pers
  "<w2>"
        <<<
        vblex inf
\end{verbatim}

Then we apply the earlier rules to our symbolic string. First two are
fine. They don't affect the analyses that are crucial to our examined
rule. However, the third SELECT rule has \texttt{vblex inf} in its
context. When we add this clause to SAT solver, it will try to find
another model which satisfies the examined rule, but will make the
SELECT rule to have no effect. We find one, which satisfies scenario 2:
condition is not met.

\begin{verbatim}
  "<w1>"
        prn pers
  "<w2>"
        <<<
        inf
\end{verbatim}

\begin{center}\rule{3in}{0.4pt}\end{center}

An example of scenario 4:

Last rule:

\begin{verbatim}
  SELECT:s_pr_v pr  + "te" "om te"  IF (1 vblex vbmod vaux vbhaver vbser  + inf ) 
\end{verbatim}

Earlier rules:

\begin{verbatim}
   REMOVE:r_pr_v pr  IF (1 vblex vbmod vaux vbhaver vbser ) (NOT 0 "te" "om te" )
   REMOVE:r_adv_n adv  IF (1 n np )
\end{verbatim}

One solution for the initial constraints:

\begin{verbatim}
  "<w1>"
        >>>
        pr "te"
  "<w2>"
        vblex inf
\end{verbatim}

First REMOVE rule conflicts. Solve by applying scenario 4: only targets
are left. Scenario 3 wouldn't work because \texttt{(pr "te")} is a
requirement. Scenario 2 wouldn't work because ???

Solution after all clauses:

\begin{verbatim}
"<w1>"
        pr "te"
        pr "om te"
"<w2>"
        vblex inf
\end{verbatim}

\subsection{Interaction of previous
rules}\label{interaction-of-previous-rules}

Look at the following three rules.

\begin{verbatim}
r1 = REMOVE V   IF (-1C Det) ;
r2 = REMOVE Det IF ( 1  V)   ;
r3 = REMOVE V   IF (-1  Det) ;
\end{verbatim}

Is there an input which can go through the rules and trigger at the
last?

\begin{verbatim}
"<w1>"
        det
        pron
"<w2>"
        v
        n
\end{verbatim}

\begin{itemize}
\itemsep1pt\parskip0pt\parsep0pt
\item
  r1: Doesn't trigger because condition \texttt{(-1C Det)} isn't met.
\item
  r2: Cannot trigger, because r3 requires \texttt{w2} to have a
  \texttt{v} analysis.
\item
  Tries to go for ``ok what if the target (\texttt{w1}) \emph{only} has
  a \texttt{det} analysis!''
\item
  In that case, the input would already trigger \texttt{r1}
\item
  Cannot do neither =\textgreater{} conflict
\end{itemize}
