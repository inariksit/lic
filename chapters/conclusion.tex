\chapter{Conclusions}
\label{chapterConclusions}

We set out to express CG as a SAT-problem. 
At the first attempt, we ended up reimplementing FSIG \cite{koskenniemi90}, 
also known as parallel CG \cite{koskenniemi97}. 
A straight-forward translation of a sequential scheme, introduced in \cite{lager_nivre01}, 
turned out not to be a SAT-problem: all that is needed is manipulation of Boolean expressions, 
without any uncertainty, or need for search. 
However, we modified the problem in a crucial detail: instead of an actual sentence, 
we applied the rules to a {\em symbolic sentence}, where every word contains every possible reading.
Then, instead of ``Which readings are true after all rule applications?'', the SAT-problem became ``Which readings were originally true?''
We exploited this property for grammar analysis: for each rule $r$, we tried to create a sentence 
which has passed through all previous rules $R$, and could still trigger $r$.
If such sentence could not be found, this means that some of the rules in $R$ made $r$ impossible to apply: for instance, by removing the same target with less conditions, or removing something from the context of $r$.
This method was successful in finding such conflicts, when tested with grammars between 70--1000 rules.

The previous paragraph summarises the past year of research, which led to this thesis. 
For the rest of this section, we broaden the view: we discuss the implications of our work, 
present new questions raised by our results, and try to find a spot for our contribution 
in the grand scheme of CG research.


\section{Contributions of this thesis}

Let us refer back to \cite{lager_nivre01}. 
Before starting their logical reconstruction of four different methods for POS tagging,
they give a list of motivations why one would want to reconstruct a method in the first place.
Below we quote a three of them, which we feel that apply to our work.

\begin{quote}
\begin{itemize}
\item It allows us---even forces us---to clarify and make more precise a formulation of a method [$M$] which is vague and open to interpretation.
\item It may put us in a position to find novel implementation techniques for $M$. 
In particular, $T_M$ in combination with a theorem prover implementing $I_M$ may turn out to be a perfectly practical implementation of $M$.
\item By factoring out the knowledge component $T_M$ from a method $M$, 
we may be able to find interesting novel uses for $T_M$ which $M$ was not designed for in the first place.
\end{itemize}
\end{quote}


\paragraph{Clarify and make more precise} This point hits very close to home.
The ``accidentally implemented FSIG'' theme, which has been present throughout this thesis, was not chosen just for narrative purposes; it reflects quite accurately the course of events when starting to implement SAT-CG.
We started out with the observation that CG rules resemble logical formulas; {\sc remove} and {\sc if } can be expressed as negations and implications respectively. 
Translating the rules into implications is fairly straightforward, but a SAT-encoding cannot go further without implementation details, such as rule ordering and execution type. 
%In other words, we noticed that just applying the rules and solving, we implement a parallel CG.
% A single grammar expresses actions to cohorts, conditional on other cohorts in specified places; but the rules alone leave very much open to interpretation.
% A grammar on its own does not specify rule ordering nor execution type; with strict vs. heuristic on one axis, and sequential vs. parallel on other. 

Of course, ``translating CG rules into logical formulas, without a notion of order, corresponds to parallel CG'' is not a unique discovery---the referred article was published in 2001, and had we read it before starting our work, none of these properties would not have surprised us.
What is, then, our contribution in clarifying and making something more precise?
The answer is in Table~\ref{table:nelikentta}, which shows different CG systems 
placed in a table, with rule order (strict vs. heuristic) on one axis, 
and execution type (sequential vs. parallel) on other.
Previously, all parallel systems assumed perfect interplay of rules, 
and there was no way of discussing ordering \todo{is this true?}. 
With our two methods of conflict handling in Chapter~\ref{chapterCGSAT}, 
we have brought the notion of strict and heuristic also to parallel execution. 

\def\satcgMax{SAT-CG\textsubscript{Max}}
\def\satcgOrd{SAT-CG\textsubscript{Ord}}

\begin{table}[h]
\centering

  \begin{tabular}{r | c | c |@{~~...~}| p{3cm}}
           & \textbf{Strict} & \textbf{Heuristic} & \textbf{Unordered} \\ \hline
\textbf{Sequential}
           & CG-1            & CG-2  % (Tapanainen1996, p. 34
           							              & --           \\ 
           & VISL CG-3       \\ 
           & Peltonen2011    \\ %Peltonen2011, p. 80
           & Yli-Jyrä2011    \\ \hline
\textbf{Parallel}
		   & \satcgOrd       & \satcgMax          & FSIG \\ 
		   &                 &                    & (SAT-CG, 
		   											when no conflicts) \\
  \end{tabular}
  \label{table:nelikentta}
  \caption{All combinations of rule ordering and execution type. \todo{Fill the table with other implementations}}
\end{table}

\paragraph{Novel implementation techniques} Arguably the most obvious point---we have implemented a CG engine using a SAT-solver. 
As we motivate in Section~\ref{sec:SAT-intro}, it is a smart idea to formulate a problem in SAT. 
Parallel CG is an instance of a more abstract search problem; there is a whole research area that tries to optimise search. Hence it pays off to formulate a specific search problem as an instance of SAT-problem, and reap the benefits of decades of optimisation.
%Whereas \cite{lager_nivre01} talk about the benefits of formulating a problem in logic---better understanding, fresh ideas by looking it from another angle, SAT-community has 

\paragraph{Novel uses for the knowledge component of CG} 

\todo{We may be interpreting this point too freely, but the point about generation fits here.
The ``knowledge component'' would be all the bits of language description in rules; like, word boundaries don't appear in the middle of rules. Going to say something more sensible of this point later.}

\section{Open questions and future work}

Philosophical questions: can CG rules {\em generate} language?
When the sentence starts off as anything, the rules become truly declarative.
We do not remove the verb after determiner, we prevent such sequence from being created in the first place.
Of course, the rules are not written to generate text, but to remove ambiguities from a limited and predictable set of alternatives.
However, some tendencies show already in the symbolic sentences generated in Chapter~\ref{chapterCGana}: for instance, the token for end of sentence (\t{<<<}) appears 
(nearly\footnote{Word boundaries may be combined in the same set with other tags, such as commas: \t{SET~Boundary~=~>>>~OR~<<<~OR~","~OR~";"}. This set may then appear also in the middle of other conditions.}) 
always as the rightmost condition.
Thus, it is expected that word boundaries are not usually generated in the middle of the symbolic sentence.

Check if the rules already enforce some trivial stuff, like not generating word boundaries in the middle of the symbolic sentence.