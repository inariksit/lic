\chapter{CG as a SAT problem}
\ref{chapterCGSAT}

\section{Introduction}

In this chapter, we present CG %\footnote{Roughly compatible with CG-2}
as a Boolean satisfiability (SAT) problem, and describe an implementation
using a SAT solver. This is attractive for several reasons: formal logic is
well-studied, and serves as an abstract language to reason about the
properties of CG. Constraint rules encoded in logic capture richer
dependencies between the tags than standard CG.

Applying logic to reductionist grammars has been explored earlier by \cite{lager98,lager_nivre01}, but it was never adopted for use.
% ; logic programming was too slow to be used for tagging or parsing. 
Since those works, SAT solving techniques have improved significantly \cite{marques_silva2010}, and they are used in domains such as microprocessor design and computational 
biology---these problems easily match or exceed CG in complexity. 
Thanks to these advances, we were able to revisit the idea and develop it
further. 

Our work is primarily inspired by \cite{lager98}, which presents constraint
rules as a disjunctive logic program, and \cite{lager_nivre01}, which
reconstructs four different formalisms in first-order logic.
Other works combining logic to CG include
\cite{lindberg_eineborg98ilp} and \cite{asfrent14}, both using Inductive Logic Programming to learn CG rules from a tagged corpus.
%We explore different approaches to ordering. 
%In the case of unordered rules, it resembles Finite-State Intersection Grammar \cite{koskenniemi90}.
%See also \cite{lager01transformation} for discussion of ordering of the CG rules.

\subsection{Previous work: Encoding in logic}\label{encoding-in-logic}

\cite{lager98} represents a CG-like, shallow and reductionist system in
logic. \cite{lager_nivre01} builds on that in a study which reconstructs
four formalisms in logic. CG is contrasted with Finite-State
Intersection Grammar (FSIG) and Brill tagging; all three work on a set
of constraint rules which modify the initially ambiguous input, but with some crucial
differences.

The rules and analyses are represented as clauses, and if it is
possible to build a model which satisfies all clauses, then there is
an analysis for the sentence.
Take the unordered case first. The authors define predicates \emph{word}
and \emph{\sc{pos}}, and form clauses as follows. The variable $P$ is used to
denote any word, and the index $n$ its position.

\begin{align*}
\text{word(P, the)} \  &\Rightarrow \ \text{{\sc pos}(P, det)} \\
\text{word(P, bear)} \ &\Rightarrow \ \text{{\sc pos}(P, verb)} \vee \text{{\sc pos}(P, noun)} \\
\text{{\sc pos}(P$_n$, det)} \wedge \text{{\sc pos}(P$_{n+1}$, verb)} \ &\Rightarrow 
\end{align*}

\noindent The first clauses read as ``if the word is \emph{the}, it
is a determiner'' and ``if the word is \emph{bear}, it can be a verb or a noun''.
The third clause represents the rule which prohibits a verb after a
determiner. It normalises to $\neg \text{{\sc pos}(P$_n$, det)} \vee \neg \text{{\sc pos}(P$_{n+1}$, verb)}$, and we know that {\sc pos}(P, det) must be true for
the word \emph{the}, thus the verb analysis for \emph{bear} must be
false.

This representation models FSIG, where the rules are logically
unordered. For CG, the authors introduce a new predicate for each rule,
$\text{{\sc pos}}^i$, where $i$ indicates the index of the rule in the
sequence of all rules.
Each rule is translated into two
clauses: 1) the conditions hold, the target has >1 analysis\footnote{In \cite{listenmaa_claessen2015}, the default rules
  are given to the SAT solver as separate clauses; for every word
  $w$, at least one of its analyses \{$w_1$,$w_2$,...,$w_n$\} must be
  true. Then the clauses for each rule don't need to repeat the ``only
  if it doesn't remove the last reading'' condition.}, and the 
targetet reading is selected or removed; and
2) the conditions don't hold or the target has only one analysis, and the target is left untouched. The general form of the clauses is shown below:

\begin{align*}
\text{{\sc pos}}^i\text{(P, T)} \: \wedge \: (\;\:\,  \text{conditions\_hold}
\wedge  \text{|T|} > 1) \  &\Rightarrow \  \text{{\sc pos}}^{i+1}\text{(P, T$\,\setminus\,$[target])} \\
\text{{\sc pos}}^i\text{(P, T)} \: \wedge \: (\neg \text{conditions\_hold} \vee
\text{|T|} = 1) \  &\Rightarrow \ \text{{\sc pos}}^{i+1}\text{(P, T)}
\end{align*}


To show a concrete example, the following shows the two rules in the specified order:
\begin{inparaenum}[\itshape a\upshape)]
\def\labelenumi{\arabic{enumi}.}
\itemsep1pt\parskip0pt\parsep0pt
\item \texttt{REMOVE verb IF (-1 det)}; and
\item  \texttt{REMOVE noun IF (-1 noun)}
\end{inparaenum}.

\begin{align*}
\text{{\sc pos}$^1$(P, [det])} \  &\Leftarrow \  \text{word(P, the)} \\
\text{{\sc pos}$^1$(P, [verb,noun])} \ &\Leftarrow \ \text{word(P, bear)} \\
\text{{\sc pos}$^1$(P, [verb,noun])} \ &\Leftarrow \ \text{word(P, sleeps)} \\ \\
\text{{\sc pos}$^2$(P$_n$, T$\,\setminus\,$[verb])} \ &\Leftarrow \ \text{{\sc pos}$^1$(P$_n$, T)}
 \wedge (\;\; \text{{\sc pos}$^1$(P$_{n-1}$, [det])}  \wedge  \text{T$\,\setminus\,$[verb] $\neq$ []}) \\
\text{{\sc pos}$^2$(P$_n$, T)} \ &\Leftarrow \  \text{{\sc pos}$^1$(P$_n$, T)} \wedge
(\neg \text{{\sc pos}$^1$(P$_{n-1}$, [det])} \vee \text{T$\,\setminus\,$[verb]
  $=$ []}) \\ \\
\text{{\sc pos}$^3$(P$_n$, T$\,\setminus\,$[noun])} \ &\Leftarrow \ \text{{\sc pos}$^2$(P$_n$, T)}
 \wedge (\;\; \text{{\sc pos}$^2$(P$_{n-1}$, [noun])}  \wedge  \text{T$\,\setminus\,$[noun] $\neq$ []}) \\
\text{{\sc pos}$^3$(P$_n$, T)} \ &\Leftarrow \  \text{{\sc pos}$^2$(P$_n$, T)} \wedge
(\neg \text{{\sc pos}$^2$(P$_{n-1}$, [noun])} \vee \text{T$\,\setminus\,$[noun]
  $=$ []})
\end{align*}

The logical reconstruction helps to provide some clarity when comparing CG
to FSIG. In FSIG, the predicate ${\sc pos}$ is a statement of an analysis of a
word; in case of uncertainty, disjunction is used to present all possible analyses. In CG, uncertainty is modelled by sets of analyses,
and the predicate ${\sc pos}^i$ is a statement of the
set of analyses of a word at a given stage of the rule sequence. The
final result is obtained by composition of these clauses in the order of
the rule sequence.



\section{CG as a SAT problem}
Let us demonstrate our approach with the following example in Spanish.

\begin{verbatim}
"<la>"
        "el" det def f sg
        "lo" prn p3 f sg
"<casa>"
        "casa" n f sg
        "casar" v pri p3 sg
        "casar" v imp p2 sg
\end{verbatim}

The ambiguous passage can be either a noun phrase, \emph{la}\texttt{<det>} \emph{casa}\texttt{<n>} 
`the house'  or a verb phrase \emph{la}\texttt{<prn>}  \emph{casa}\texttt{<v><pri><p3>} `(he/she) marries her'. 
We add the following rules:

\begin{itemize}
\item [] \texttt{REMOVE prn IF (1 n) ;} \\
             \texttt{REMOVE det IF (1 v) ;}
\end{itemize}

Standard CG will apply one  of the rules to the word \emph{la}; 
either the one that comes first, or by some other heuristic. 
The other rule will not fire, because it would remove the last
reading. 
If we use the cautious mode (\texttt{1C n} or \texttt{1C v}), which
 requires the word in the context to be fully disambiguated, 
neither of the rules will be applied.
In any case, all readings of \emph{casa} are left untouched by these rules.

The SAT solver performs a search, 
and starts building possible models that satisfy both constraints. 
In addition to the given constraints, we have default rules to emulate
the CG principles: an analysis is true if no rule affects it,
and at least one analysis for each word is true---the notion of
``last'' is not applicable.

With these constraints, we get two solutions. The interaction of the
rules regarding \emph{la}  disambiguates the part of speech of
\emph{casa} for free, and the order of the rules does not matter. 

\begin{enumerate}
\item [\texttt{1)}]
\begin{verbatim}
"<la>"
        "el" det def f sg
"<casa>"
        "casa" n f sg
\end{verbatim}
\item [\texttt{2)}]
\begin{verbatim}
"<la>"
        "lo" prn p3 f sg
"<casa>"
        "casar" v pri p3 sg
        "casar" v imp p2 sg
\end{verbatim}
\end{enumerate} 

% Without more context or additional rules we cannot fully disambiguate the passage,
% but unlike with standard CG, we get only legal combinations in one solution.



\noindent The most important differences between the traditional and
the SAT-based approach are described in the following sections.

\subsection{Rules disambiguate more}
Considering our example phrase and rules, the standard CG implementation
can only remove readings from the target word (\texttt{prn} or
\texttt{det}). The SAT-based implementation interprets the rules as
``determiner and verb together are illegal'', and is free to take action that concerns also the word in the condition (\texttt{n} or \texttt{v}).

This behaviour is explained by simple properties of logical formulae.
When the rules are applied to the text, they are translated into
implications: \texttt{REMOVE prn IF (1 n)} becomes
 \emph{casa}\texttt{<n>} $\Rightarrow$ $\neg$\emph{la}\texttt{<prn>},
 which reads ``if the \texttt{n} reading for \emph{casa} is true, then
 discard the \texttt{prn} reading for \emph{la}''.
Any implication $a\,\Rightarrow\,b$ can be represented as a disjunction
$\neg a\,\vee\,b$; intuitively, either the antecedent is false
and the consequent can be anything, or the consequent is true and the
antecedent can be anything.
Due to this property, our rule translates into the disjunction 
$\neg$\emph{casa}\texttt{<n>} $\vee$ $\neg$\emph{la}\texttt{<prn>},
which is also equivalent to another implication, 
 \emph{la}\texttt{<prn>} $\Rightarrow$ $\neg$\emph{casa}\texttt{<n>}.
This means that the rules are logically flipped: \texttt{REMOVE prn IF
  (1 n)} translates into the same logical formula as  \texttt{REMOVE n
  IF (-1 prn)}. 
A rule with more conditions corresponds to many rules, each condition
taking its turn to be the target. % of removal or selection.

% \begin{itemize}
% \item [] \texttt{REMOVE n IF (-1 prn) ;} \\
%          \texttt{REMOVE v IF (-1 det) ;}
% \end{itemize}

% SAT-based approach gives identical results with both sets of rules,
% whereas the standard CG would remove one reading from \emph{casa} and leave \emph{la} ambiguous.
% Testing this property with more complex rules and larger rule sets remains to be done.

%Rules with more conditions translate into many rules; rules with negation become complements (\texttt{(*)-X} for \texttt{NOT X}). Rules which require the condition to be unambiguously tagged, don't have an equivalent flip in the standard CG.
% We did not expect much practical benefits, save for realising that some rules are bad by having to think further what it implies,
%Out of interest, we manually flipped a small grammar and tested it against the original.
% \todo{The results show  }
%\begin{itemize}
%\item []          \texttt{SL Inf IF (-2 V-MOD) (-1 "not");} \\
%$\rightarrow$\texttt{SL V-MOD   IF (1 "not") (2 INF);} \\
%$\rightarrow$\texttt{SL "not" IF (-1 V-MOD) (1 Inf);}
%\item []           \texttt{SL Foo IF (NOT 1 Bar)} \\
%$\rightarrow$\texttt{SL (*)-Foo if (-1 Bar)}
%\end{itemize}


\subsection{Cautious context is irrelevant}
% Rather than waiting for a word to get disambiguated, the SAT solver starts by 
% making assumptions (e.g. ``\emph{casa} is a noun'') and working under them,
% discarding the assumption if it doesn't lead to a model that satisfies
% all constraints.

Traditional CG applies the rule set iteratively:
some rules fire during the first iteration, either because their
conditions do not require cautious context, or because some words are
unambiguous to start with. This makes some more words unambiguous, and
new rules can fire during the second iteration.

In SAT-CG, the notion of cautious context is irrelevant. Instead of
removing readings immediately,  each rule generates a  number of
implications, and the SAT solver tries to find a model that will satisfy them. 


Let us continue with the earlier example. 
We can add a word to the input:
\begin{itemize}
\item [] \emph{la casa grande} `the big house'
\end{itemize}
and a rule that removes verb reading, if the word is followed by an adjective:
\begin{itemize}
\item [] \texttt{REMOVE v IF (1 adj) ;}
\end{itemize}

% The SAT solver has already the clauses produced by the previous rules.
% \emph{casa}\texttt{<n>} $\Rightarrow$ $\neg$\emph{la}\texttt{<prn>}
% and   \emph{casa}\texttt{<v>} $\Rightarrow$ $\neg$\emph{la}\texttt{<det>}
The new rule adds the implication
 \emph{grande}\texttt{<adj>} $\Rightarrow$
 $\neg$\emph{casa}\texttt{<v>}, which will
disambiguate  \emph{casa} to a noun\footnote{Assuming that
  \texttt{adj} is the only reading for \emph{grande}, it must be true,
  because of the restriction that at least one analysis for each word
  is true. Then the implication has a true antecedent (\emph{grande}\texttt{<adj>}), thus its
  consequent ($\neg$\emph{casa}\texttt{<v>}) will hold.}. 
As the status of \emph{casa} is resolved, the SAT solver can now
discard the model where \emph{casa} is a verb and \emph{la} is a
pronoun
%---all these decisions are connected---
and we get a unique solution with \texttt{det n adj}.
% Now \emph{casa}\texttt{<n>} is true, so \emph{la}\texttt{<prn>} must be false. 
%Because \emph{la}\texttt{<prn>} is false,  \emph{casa} cannot
% be a \texttt{v}, and we get a unique solution with \texttt{det n adj}.

Contrast this with the behaviour of the standard CG.
With the new rule, standard CG will also remove the verb reading from \emph{casa}, 
but it is in no way connected to the choice for \emph{la}. It all
depends on the order of the two rules; if the \texttt{det} reading of
\emph{la} is removed first, then we are stuck with that choice.
If we made the first rules cautious, that is, keeping the determiner
open until \emph{casa} is disambiguated, then we get the same
result as with the SAT solver.
Ideally, both ways of grammar writing should yield similar results;
traditional CG rules are more imperative, and SAT-CG
rules are more declarative.

% The SAT-based approach only removes readings after it has enough
% evidence to do that. From the grammar writer perspective, this removes
% a burden of having to decide whether the rule should be cautious or
% not---the SAT solver will only take action the surrounding context
% supports the decision.

% \todo{Unordered rules: also not applicable anymore to apply rules
% iteratively; there's no ``this rule doesn't fire now but will after
% applying X and Y'', it's all just implications ``this rule will fire
% if this is true'' and let the SAT solver find if those rules can apply
% peacefully to the same input.}


\subsection{Rules can be unordered}
\label{sec:unord}

As hinted by the previous property, the SAT solver does not need a fixed
order of the rules.
Applying a rule to a sentence produces a number of clauses,
and those clauses are fed into the SAT solver.
However, in the unordered scheme, some information is lost: the
following rule sets would be treated identically, whereas in the
traditional CG, only the first would be considered as a bad order.

\begin{itemize}
\item [\texttt{1)}] \texttt{SELECT v ;} \\
         \texttt{REMOVE v IF (-1 det) ;} 
\item [\texttt{2)}] \texttt{REMOVE v IF (-1 det) ;} \\
         \texttt{SELECT v ;}
\end{itemize}

Without order, both of these rule sets will conflict, if applied to an
input that has sequence \texttt{det v}.
The SAT solver is given clauses that tell to select a verb and remove a
verb, and it cannot build a model that satisfies all of those clauses.
To solve this problem, we create a variable for every instance of rule application, and request a solution where maximally many of these variables are true.
If there is no conflict, then the maximal solution is one where all of
these variables are true; that is, all rules take action.


In case of a conflict, the SAT solver makes it possible to discard only
minimal amount of rule applications. Continuing with the example,
it is not clear which instances would be discarded, but if the rules
were part of a larger rule set, and in the context the \textsc{remove} rule was
the right one to choose, it is likely that the interaction between the desired
rules would make a large set of clauses that fit together, and the
\textsc{select} rule would not fit in, hence it would be discarded. 

This corresponds loosely to the common design pattern in
CGs, where there is a number of rules with the same target, ordered
such that more secure rules come first,
with a catch-all rule with no condition as the last resort, to be
applied if none of the previous has fired.
The order-based heuristic in the traditional CG is replaced by a more
holistic behaviour: if the rules conflict, discard the one that seems
like an outlier.


We can also emulate order with SAT-CG. To do that, we enter clauses
produced by each rule one by one, and assume the solver state reached
so far is correct. If a new clause introduces a conflict with
previous clauses, we discard it and move on to the next rule.
By testing against gold standard, we see that this scheme works better
with ready-made CGs, which are written with ordering in mind.
It also runs slightly faster than the unordered version.
\\



\noindent These three features influence the way rules are written. 
We predict that less rules are needed; whether this holds in the order
of thousands of rules remains to be tested. On the one hand, getting rid
of ordering and cautious context could ease the task of the grammar
writer, since it removes the burden of estimating the best sequence of
rules and whether to make them cautious. On the other
hand, lack of order can make the rules less transparent, and might not scale up for larger grammars.


\section{Evaluation}
\label{sec:eval}

For evaluation, we measure the performance against the state-of-the-art CG parser VISL CG-3.
% both in terms of accuracy and execution time.
SAT-CG fares slightly worse for accuracy, and significantly worse for execution time.
The results are presented in more detail in the following sections.

%This suggests that our implementation should not compete with existing
%state-of-the-art, but rather it has value as a way of relating the CG
%formalism to wider context in the theory of computer science.
%In Section~\ref{sec:apps} we discuss more about possible applications.



\subsection{Performance against VISL CG-3}


We took a manually tagged
corpus\footnote{\url{https://svn.code.sf.net/p/apertium/svn/branches/apertium-swpost/apertium-en-es/es-tagger-data/es.tagged}}
containing approximately 22,000 words of Spanish news text, 
and a small constraint grammar\footnote{\url{https://svn.code.sf.net/p/apertium/svn/languages/apertium-spa/apertium-spa.spa.rlx}}, produced independently of the authors.
% We ignored substitute rules, but kept subreadings and unification, even though our implementation doesn't handle them.
We kept only \textsc{select} and \textsc{remove} rules, which left us 261 rules.
With this setup, we produced an ambiguous version of the tagged
corpus, and ran both SAT-CG and VISL CG-3 on it.
Treating the original corpus as the gold standard, the disambiguation
by  VISL CG-3 achieves F-score of 82.6 \%, ordered SAT-CG 81.5 \%  and
unordered SAT-CG 79.2 \%. 
We did not test with other languages or text genres due to the lack of
available gold standard.
%This result suggests that that when running grammars that are written
%with the traditional CG in mind, SAT-CG loses with both ordering
%strategies, but emulating order fares better.

We also tested whether SAT-CG outperforms traditional CG with a
small rule set. With our best performing and most concise
grammar\footnote{\url{https://github.com/inariksit/cgsat/blob/master/data/spa\_smallset.rlx}}
of only 19 rules, both SAT-CG and VISL CG-3  achieve a F-score of
around 85 \%. This experiment is very small and might be explained by
overfitting or mere chance, but it seems to indicate that rules that
work well with SAT-CG are also good for traditional CG.
% It was possible to attain a rule set for which the unordered SAT-CG
% works better (\textless0.5 \%), but it required trying
% subsequences of the 19-rule set in a brute force manner---this hardly
% presents any real life use case.



% Similar patterns were observed with small (\textless{}20) rule sets
% written by the authors; depending on the subset, SAT-CG and VISL CG-3 had
% a difference of at most $\Mypm$ 1.5 \%. 
% Introducing rules one by one up to 19, the
% performance improved in a very similar rate, with less than 0.5 \%
% difference between the systems at each new rule.
% We did not evaluate on other languages or text genres, due to lack of suitable test data.

% Additional tests could include plugging SAT-CG into Apertium
% translation pipeline, and comparing the translation quality.

\subsection{Execution time}

The worst-case complexity of SAT is exponential, whereas the standard
implementations of CG are polynomial, but with advances in SAT solving
techniques, the performance in the average case in practice is more feasible than in the previous works done in 90s--00s.
We used the open-source SAT solver MiniSat \cite{een04sat}.


We tested the performance by parsing Don Quijote (384,155 words) with
the same Spanish grammars as in the previous experiment. 
Table~\ref{table:time} shows execution times compared to VISL CG-3;
SAT-CG\textsubscript{u} is the unordered scheme and
SAT-CG\textsubscript{o} is the ordered.
% For both systems, the number of rules in the grammar affects the performance more than the word count.
From the SAT solving side, maximisation is the most costly operation. 
Emulating order is slightly faster, likely because the maximisation problems are smaller.
In any case, SAT does not seem to be the bottleneck: with 261 rules,
the maximisation function was called 147,253 times, and with 19 rules,
132,255 times, but 
%However, with both rule sets, half of the sentences in Don Quijote
%needed less than 6 calls of maximise, and 75 \% of the sentences
%needed less than 14.
the differences in the execution times are much larger, which suggests
that there are other reasons for the worse performance. 
This is to be expected, as SAT-CG is currently just a naive
proof-of-concept implementation with no optimisations.

\begin{table}
  \centering
  \begin{tabular}{|c|c|c|c|}
     \hline
     \textbf{\# rules} &  \textbf{SAT-CG\textsubscript{u}} & \textbf{SAT-CG\textsubscript{o}} & \textbf{VISL CG-3} \\ \hline
      19   & 39.7s &  22.1s   & 4.2s\\ %\hline
      99   & 1m34.1s & 1m14.9s & 6.1s \\ %\hline
      261  & 2m54.1s & 2m31.6s & 10.7s \\ \hline
  \end{tabular}
  \caption{Execution times for 384,155 words.}
  \label{table:time}
\end{table}

\section{Applications and future work}
\label{sec:apps}
%As demonstrated in Section~\ref{sec:eval}, our implementation is not
%competitive with the state of the art. 
%VISL CG-3 is efficient, reliable and widely used, and
Instead of trying to compete with the state of the art, we plan to use SAT-CG for grammar analysis\footnote{We thank Eckhard Bick for the idea.}.
There has been work on automatic tuning of hand-written CGs
\cite{bick2013tuning}, but to our knowledge no tools to
search for inconsistencies or suboptimal design. %in order to point it out to the grammar writer.

The sequential application of traditional CG rules is good for
performance and transparency. When a rule takes action, the analyses
are removed from the sentence, and the next rules get the modified
sentence as input. 
% At the execution of rule, there is no way to go back to earlier rules and undo them. 
As a downside, there is no way to know which part comes directly from the
raw input and which part from applying previous rules.

A conflict in an ordered scheme can be defined as 
a set of two or more rules, such that applying the first makes the
next rules impossible to apply, regardless of the input.
We can reuse the example from Section~\ref{sec:unord}:

\begin{itemize}
\item [] \texttt{SELECT v ;} \\
         \texttt{REMOVE v IF (-1 det) ;}
\end{itemize}

The first rule selects the verb reading everywhere and removes
all other readings, leaving no chance for the second rule to take action.
If the rules are introduced in a different order, there is no
conflict: the \textsc{remove} rule would not remove verb readings from all
possible verb analyses, so there is a possibility for the \textsc{select} rule
to fire.


Ordered SAT-CG can be used to detect these conflicts without any
modifications, as a side effect of its design. 
After applying each rule, it stores the clauses
produced by the rule and commits to them.
In case of a conflict, the program detects the particular rule that 
violates the previous clauses, with the sentence
where it is applied. Thus we get feedback which rule fails, and on which
particular word(s).

% This conflict solving can be done per section, or for the whole grammar. 
% In case of section-based application, we run each section on the
% input, and give the output as a fresh input to the new section, discarding
% all clauses. Alternatively, we can let the rules interact with each
% other regardless of the section boundaries.

Unordered SAT-CG with maximisation-based conflict solving is not
suitable for this task: the whole definition of conflict depends on
ordering, and the unordered scheme deliberately loses this information.
On a more speculative note, an unordered formalism such as Finite-State
Intersection Grammar \cite{koskenniemi90} might benefit from the
maximisation-based technique in conflict handling.

Finally, we intend to test for conflicts
without using a corpus.
Let us illustrate the idea with the same two rules,
\texttt{SELECT v} and \texttt{REMOVE v IF (-1 det)}
in both orders.
Assume we have the tag set \texttt{\{det, n, v\}}, 
and we want to find if there exists an input such that
 both rules, applied in the given order, remove something from the input.
% This is another constraint solving problem, hence we can model it in SAT.
There are no inputs that satisfy the requirement with the first order,
but several that work with the second, such as the following:

\begin{itemize}
\item []
\begin{verbatim}
"<w1>"
        det
        v
"<w2>"
        n
        v
\end{verbatim}
\end{itemize}
Thus we can say that the first rule order is conflicting, but the second
one is not. 
Implementing and testing this on a larger scale is left for future work.


% SELECT (a) IF (0 PALJON + Nom + Sg) (1 N + Nom + Sg) ; # PALJO VALITTAMINEN
% SELECT (a) IF (0 PALJON + Par) (1 N + Par + Sg) ;
% SELECT (a) IF (0 PALJON + LOC-CASE) (1 N + 1 LOC-CASE + Sg) ;
% REMOVE (a) IF (0 PALJON) ; # "on asetettu paljon haltijaksi"

\section{Conclusions}

SAT-solvers are nowadays powerful enough to be used for dealing with
Constraint Grammar. A logic-based approach to CG has possible
advantages over more traditional approaches; a SAT solver may
disambiguate more words, and may do so more precisely, capturing
more dependencies between tags.
We experimented with both ordered and unordered rules, and found the 
ordered scheme to work better with previously written grammars.
For future direction, we intend to concentrate on grammar analysis,
especially finding conflicts in constraint grammars.

\def\sobre{\text{\em sobre}}
\def\una{\text{\em una}}
\def\aproximacion{\text{ \em aproximaci\'{o}n}}
\def\mas{\text{\em m\'{a}s}}
\def\cientifica{\text{\em cient\'{\i}fica}}


\def\vPrsPThree{\text{\sc PrsP3}}
\def\vPrsPOne{\text{\sc PrsP1}}
\def\vImpPThree{\text{\sc ImpP3}}
\def\adj{{\text{\sc Adj}}}
\def\adv{{\text{\sc Adv}}}
\def\n{\text{\sc N}}
\def\pr{{\text{\sc Pr}}}
\def\prn{{\text{\sc Prn}}}
\def\det{{\text{\sc Det}}}
\def\notDet{{\neg \text{\sc Det}}}
\def\any{{\text{Any}}}


\def\sobrePr{\sobre_\pr}
\def\sobreN{\sobre_\n}

\def\unaNotDet{\una_\notDet}
\def\unaAny{\una_\any}
\def\unaPrn{\una_\prn}
\def\unaDet{\una_\det}
\def\unaPrsPThree{\una_\vPrsPThree}
\def\unaPrsPOne{\una_\vPrsPOne}
\def\unaImp{\una_\vImpPThree}
\def\aproximacionN{\aproximacion_\n}
\def\masAdv{\mas_\adv}
\def\masAdj{\mas_\adj}
\def\cientificaAdj{\cientifica_\adj}
\def\cientificaN{\cientifica_\n}


\def\newVar{\text{}} %New variable }}

\def\cgrule#1{\noindent {\bf  #1 }}
\def\eqdef{\Coloneqq}
%\def\eqdef{\Longleftrightarrow}
\def\impl{\quad\Longrightarrow\quad}

\def\ob#1{\overbrace{ #1 \rule{0pt}{2ex}}}


%%%%%%%%%%%%%%%

%\begin{verbatim}
% "<sobre>"
%         "sobre" pr
%         "sobre" n m sg
% "<una>"
%         "uno" prn tn f sg
%         "uno" det ind f sg
%         "unir" verb prs p3 sg
%         "unir" verb prs p1 sg
%         "unir" verb imp p3 sg
% "<aproximación>"
%         "aproximación" n f sg
% "<más>"
%         "más" adv
%         "más" adj mf sp
% "<científica>"
%         "científico" adj f sg
%         "científico" n f sg
%\end{verbatim}

%%%%%%%%%%%%%%%

%\chapter{SAT-encoding}

\label{appendix1}

\noindent In this appendix, we show the SAT-encoding for all the rule types that 
SAT-CG supports: 
{\sc remove} and {\sc select} rules, with the contextual tests and set operations described in CG-2 \cite{tapanainen1996}, except for the operators \t{**} (continue search if linked tests fail) and \t{@} (absolute position). In addition, the software SAT-CG supports a couple of new constructions from VISL CG-3: {\sc cbarrier}, which requires the barrier tag to be unambiguous in order to function as a barrier;
{\sc negate}, which inverts a whole chain of contextual tests; and subreadings, which make it possible to include e.g. a clitic in the same cohort, but distinguish it from the main reading.

The encoding in this appendix is independent of the implementation of the software SAT-CG; it does not describe how the sentence is processed in order to find the variables. 
The purpose of this appendix is to give a full description of the different rule types; the method is introduced with more pedagogical focus in Section~\ref{sec:CGSAT}. 
%However, the examples in this appendix are self-contained; if you know my previous stuff and want to be bored, just read this!



\section{SAT-encoding of sentences}

As in Section~\ref{sec:CGSAT}, we demonstrate the SAT-encoding with a concrete segment in Spanish:  \emph{sobre una aproximación más científica}. 
It has the virtue of being ambiguous in nearly every word: for instance, \emph{sobre} is either a preposition (`above' or `about') or a noun (`envelope'); \emph{una} can be a pronoun, determiner or a verb. The full analysis, with the initial ambiguities, is shown in Figure~\ref{fig:satEncodingSpanishExample}. 

\begin{figure}[t]
\ttfamily
\centering
\begin{tabular}{ll @{\hspace{1.5cm}} ll}
"<sobre>"  &                     &                    &                         \\ 
           & "sobre" pr          &  "<aproximación>"  &                         \\
           & "sobre" n m sg      &                    & "aproximación" n f sg   \\
"<una>"    &                     &   "<más>"          &                         \\
           & "uno" prn tn f sg   &                    & "más" adv               \\
           & "uno" det ind f sg  &                    & "más" adj mf sp         \\
           & "unir" v prs p3 sg  &  "<científica>"    &                         \\
           & "unir" v prs p1 sg  &                    & "científico" adj f sg   \\
           & "unir" v imp p3 sg  &                    & "científico" n f sg     \\

\end{tabular}
\caption{Ambiguous segment in Spanish.}
\label{fig:satEncodingSpanishExample}
\end{figure}

\noindent We transform each reading into a SAT-variable:

\begin{equation}
\begin{array}{c c c c c}
\sobrePr & \unaPrn & \aproximacionN & \masAdv & \cientificaAdj \\
\sobreN  & \unaDet &                & \masAdj & \cientificaN \\
         & \unaPrsPThree \\
         & \unaPrsPOne \\
         & \unaImp \\
\end{array}
\end{equation}

CG rules may not remove the last reading, even if the conditions hold otherwise.
To ensure that each cohort contains at least one true variable, we add the clauses in \ref{eq:atleastOneTrue}; later referred to as the ``default rule''. 
The word \emph{aproximación} is already unambiguous, thus the clause $\aproximacionN$ is a unit clause, and the respective variable is trivially assigned $True$. 
The final assignment of the other variables depends on the constraint rules.


\begin{equation}
\begin{array}{c}
\sobrePr \vee \sobreN \\
\unaPrn \vee \unaDet \vee \unaPrsPThree \vee \unaPrsPOne \vee \unaImp \\
\aproximacionN \\
\masAdv \vee \masAdj \\
\cientificaAdj \vee \cientificaN \\
\end{array}
\label{eq:atleastOneTrue}
\end{equation}


\section{SAT-encoding of rules}

In order to demonstrate the SAT-encoding, we show variants of \textsc{remove} and \textsc{select} rules, with different contextual tests. 
We try to craft rules that make sense for this segment; however, some variants are not likely encountered in a real grammar, and for some rule types, we modify the rule slightly. We believe this makes the encoding overall more readable, in contrast to using more homogenous but more artificial rules and input.

%The basis of the rule is \texttt{REMOVE adj}, which will target the word form \emph{más}.
% Its positive counterpart is \texttt{SELECT adv}, which translates into ``remove everything but adv''.

% \subsection{Parallel scheme}

% We begin by introducing the parallel scheme. For basic rules, this corresponds to the encoding in \cite{lager98}; each analysis is given a single variable, and all rules have access to it.
% and all the rules are applied at the same time. 
%In addition, we add rule types that \cite{lager98} does not consider.

\subsection{No conditions} 

The simplest rule types remove or select a target in all cases. 
A rule can target one or multiple readings in a cohort. We demonstrate the case with one target in rules~\ref{eq:rmAdj}--\ref{eq:slAdj}, and multiple target in rules~\ref{eq:rmVb}--\ref{eq:slVb}. \\


\cgrule{REMOVE adj} Unconditionally removes all readings which contain the target.

\begin{equation}
\begin{array}{c}
\neg{}\masAdj \\
\neg{}\cientificaAdj
\end{array}
\label{eq:rmAdj}
\end{equation}

\cgrule{SELECT adj} Unconditionally removes all other readings which are in the same cohort with the target.
We do not add an explicit clause for $\masAdj$ and $\cientificaAdj$: the default rule \ref{eq:atleastOneTrue} already contain $\masAdv \vee \masAdj$ and $\cientificaAdj \vee \cientificaN$, and the combination $\neg\masAdv$  and $\masAdv \vee \masAdj$  implies that $\masAdj$ must be true. 

\begin{equation}
% \begin{array}{r @{~\wedge~} l }
% \masAdj        & \neg{}\masAdv \\
% \cientificaAdj & \neg{}\cientificaN \\
% \end{array}
\begin{array}{c}
\neg{}\masAdv \\
\neg{}\cientificaN
\end{array}
\label{eq:slAdj}
\end{equation}

% \cgrule{REMOVE n} A rule may not remove the last possible reading: accordingly, $\aproximacionN$ is not negated.

% \begin{equation}
% \begin{array}{c}
% \neg{}\sobreN
% \end{array}
% \label{eq:rmN}
% \end{equation}



\cgrule{REMOVE verb} As \ref{eq:rmAdj}, but matches multiple readings in a cohort. All targets are negated.

\begin{equation}
\begin{array}{c @{~\wedge~} c @{~\wedge~} c}
\neg{} \unaPrsPThree  & \neg \unaPrsPOne & \neg \unaImp
\end{array}
\label{eq:rmVb}
\end{equation}

\cgrule{SELECT verb} As \ref{eq:slAdj}, but matches multiple readings in a cohort. All other readings in the same cohort are negated.
As previously, we need no explicit clause for $\unaPrsPThree  \vee \unaPrsPOne \vee \unaImp$: the default rule~\ref{eq:atleastOneTrue} guarantees that at least one of the target readings is true.
% In the case of multiple targets, it could actually be harmful

\begin{equation}
\begin{array}{c @{~\wedge~} c}
\neg \unaDet & \neg \unaPrn \\
%(\unaPrsPThree  \vee \unaPrsPOne \vee \unaImp) & (\neg \unaDet \wedge \neg \unaPrn) \\
\end{array}
\label{eq:slVb}
\end{equation}


\subsection{Positive conditions} 

Rules with contextual tests apply to the target, if the conditions hold. 
This is naturally represented as implications. We demonstrate both \textsc{remove} and {\sc select} rules with a single condition in \ref{eq:singleCondition}; the rest of the variants only with {\sc remove}. All rule types can be changed to {\sc select} by changing the consequent from $\neg{}\masAdj$ to $\neg{}\masAdv$. %from $\neg{}\masAdj$ to $\masAdv \wedge \neg{}\masAdj$.


\begin{equation}
\begin{array}{l @{\hspace{2.5cm}} l}
\text{\cgrule{REMOVE adj IF (1 adj)}}   &  \text{\cgrule{SELECT adj IF (1 adj)}} \\
\cientificaAdj \Longrightarrow \neg{}\masAdj &  \cientificaAdj \Longrightarrow  \neg{}\masAdv \\
\end{array}
\label{eq:singleCondition}
\end{equation}

%
% \intertext{\cgrule{REMOVE adj IF (1 adj)} Single condition, {\sc remove} variant.} 
% \cientificaAdj & \impl  \neg{}\masAdj \\
% %
% %
% \intertext{\cgrule{SELECT adv IF (1 adj)} Single condition, {\sc select} variant. Target is selected, and everything else in the reading is removed.} 
% \cientificaAdj & \impl \masAdv \wedge \neg{}\masAdj \\
%
%
\begin{align}
\intertext{\cgrule{REMOVE adj IF (-1 n) (1 adj)} Conjunction of conditions.}
\intertext{\cgrule{REMOVE adj IF (-1 n LINK 2 adj)} Linked conditions---identical to the above.}
\cientificaAdj \wedge \aproximacionN & \impl \neg{}\masAdj \\
%
%
\intertext{\cgrule{REMOVE adj IF ((-1 n) OR (1 adj))} Disjunction of conditions (template).}
\cientificaAdj \vee \aproximacionN & \impl  \neg{}\masAdj \\
%
%
\intertext{\cgrule{REMOVE adj IF (1C adj)} Careful context. Condition must be must be unambiguously adjective.}
\cientificaAdj \wedge \neg{}\cientificaN & \impl \neg{}\masAdj \\
%
%
\intertext{\cgrule{REMOVE adj IF (-1* n)} Scanning. Any noun before the target is a valid condition.}
%
\sobreN \vee \aproximacionN & \impl  \neg{}\masAdj \notag \\
\sobreN \vee \aproximacionN & \impl  \neg{}\cientificaAdj \\
%
%
\intertext{\cgrule{REMOVE adj IF (-1* n LINK 1 v)} Scanning and linked condition. Any noun before the target, followed immediately by a verb, is a valid condition.}
%
\sobreN \wedge (\unaPrsPThree  \vee \unaPrsPOne \vee \unaImp) & \impl  \neg{}\masAdj \notag \\
\sobreN \wedge (\unaPrsPThree  \vee \unaPrsPOne \vee \unaImp) & \impl  \neg{}\cientificaAdj \\
%
%
\intertext{\cgrule{REMOVE adj IF (-1* n LINK 1* adv)} Scanning and linked condition. Any noun before the target, followed anywhere by an adverb, is a valid condition.}
%
(\sobreN \wedge \masAdv) \vee (\aproximacionN \wedge \masAdv) & \impl  \neg{}\masAdj \notag \\
(\sobreN \wedge \masAdv) \vee (\aproximacionN \wedge \masAdv) & \impl  \neg{}\cientificaAdj \\
%
%
\intertext{\cgrule{REMOVE adj IF (-1* n BARRIER det)} Scanning up to a barrier. Any noun before the target, up to a determiner, is a valid condition:
%The noun reading of {\em sobre} is not a valid condition, if the determiner reading of {\em una} holds.
$\sobreN$ is a valid condition only if $\unaDet$ is false.}
%
(\sobreN \wedge \neg \unaDet) \vee \aproximacionN & \impl \neg{}\masAdj \notag \\
(\sobreN \wedge \neg \unaDet) \vee \aproximacionN & \impl \neg{}\cientificaAdj 
%
\intertext{\cgrule{REMOVE adj IF (-1* n CBARRIER det)} Scanning up to a careful barrier.
Any noun before the target, up to an unambiguous determiner, is a valid condition.
The variable $\unaDet$ fails to work as a barrier, if any of the other analyses of $\una$ is true.
Let $\unaAny$ denote the disjunction $\unaPrn \vee \unaPrsPThree \vee \unaPrsPOne \vee \unaImp$.} %Then we write the condition as follows.}
(\sobreN \wedge \unaAny) \vee \aproximacionN & \impl \neg{}\masAdj \notag \\
(\sobreN \wedge \unaAny) \vee \aproximacionN & \impl \neg{}\cientificaAdj
%
%
%
\intertext{\cgrule{REMOVE adj IF (-1C* n)} Scanning with careful context. Any unambiguous noun before the target is a valid condition: $\sobreN$ is a valid condition only if it is the only reading in its cohort, i.e. $\sobrePr$ is false.}
%
(\sobreN \wedge \neg \sobrePr) \vee \aproximacionN & \impl  \neg{}\masAdj \notag \\
(\sobreN \wedge \neg \sobrePr) \vee \aproximacionN & \impl  \neg{}\cientificaAdj 
%
%
\intertext{\cgrule{REMOVE adj IF (-1C* n BARRIER det)} Scanning with careful context, up to a barrier.
Any unambiguous noun before the target, up to a determiner, is a valid condition:
$\sobreN$ is a valid condition only if it is the only reading in its cohort ($\sobrePr$ is false), and there is no determiner between \sobre{} and the target ($\unaDet{}$ is false).}
% $\sobreN$ is a valid condition only if it is the only reading, and $\una$ is not a determiner.}
%
%
(\sobreN \wedge \neg \sobrePr \wedge \neg \unaDet) \vee \aproximacionN & \impl \neg{}\masAdj \notag \\
(\sobreN \wedge \neg \sobrePr \wedge \neg \unaDet) \vee \aproximacionN & \impl \neg{}\cientificaAdj 
%
\intertext{\cgrule{REMOVE adj IF (-1C* n CBARRIER det)} Scanning with careful context, up to a careful barrier. Like above, but \una{} fails to work as a barrier if it is not unambiguously determiner.}
%
%
(\sobreN \wedge \neg \sobrePr \wedge \unaAny) \vee \aproximacionN & \impl \neg{}\masAdj \notag \\
(\sobreN \wedge \neg \sobrePr \wedge \unaAny) \vee \aproximacionN & \impl \neg{}\cientificaAdj 
\end{align}


\subsection{Inverted conditions}

In the following, we demonstrate the effect of two inversion operators. 
The keyword {\sc not} inverts a single contextual test, such as \cgrule{IF (NOT 1 noun)}, as well as linked conditions, such as \cgrule{IF (-2 det LINK NOT *1 noun)}. The keyword {\sc negate} inverts a whole conjunction of contextual tests, which may have any polarity: \cgrule{IF (NEGATE -2 det LINK NOT 1 noun)} means ``there may not be a determiner followed by a not-noun''; thus, ``det noun'' would be fine, or ``prn adj'', but not ``det adj''. %a sequence which matches the whole``
Inversion cannot be applied to a {\sc barrier} condition. If one wants to express \cgrule{IF (*1 foo BARRIER $\neg$bar)}, that is, ``try to find a \emph{foo} until you see the first item that is not \emph{bar}'', a set complement operator must be used: \cgrule{(*) - bar}.

There is a crucial difference between matching positive and inverted conditions.
If a positive condition is out of scope or the tag is not present in the initial analysis, the rule simply does not match, and no clauses are created. For instance, the conditions `10 adj' or `-1 punct', matched against our example passage, would not result in any action.
In contrast, when an inverted condition is out of scope or unapplicable, 
that makes the action happen unconditionally.
As per VISL CG-3, the condition \cgrule{NOT 10 adj} applies to all sentences where there is no 10th word from target that is adjective; including the case where there is no 10th word at all.
If we need to actually have a 10th word to the right, but that word may not be an adjective, we can, again, use the set complement: \cgrule{IF (10 (*) - adj)}.





\begin{align}
\intertext{\cgrule{REMOVE adj IF (NOT 1 adj)} Single inverted condition. 
There is no word following \cientifica{}, hence its adjective reading is removed unconditionally.}
\neg \cientificaAdj & \impl  \neg{}\masAdj \notag \\
                    & \impl \neg \cientificaAdj \\
%
\intertext{\cgrule{REMOVE adj IF (NOT -1 n) (NOT 1 adj)} Conjunction of inverted conditions.}
 \neg \aproximacionN \wedge \neg \cientificaAdj & \impl \neg{}\masAdj \notag \\
                                                & \impl \neg \cientificaAdj
%
%
\intertext{\cgrule{REMOVE adj IF (NEGATE -3 pr LINK 1 det LINK 1 n)} Inversion of a conjunction of conditions.}
%
\neg (\sobrePr \wedge \unaDet \wedge \aproximacionN) & \impl \neg{}\masAdj \notag \\
                                                     & \impl \neg \cientificaAdj \\
%
%
\intertext{\cgrule{REMOVE adj IF (NOT 1C adj)} Negated careful context. Condition cannot be unambiguously adjective.}
%
\neg (\cientificaAdj \wedge \neg \cientificaN) & \impl \neg{}\masAdj \notag \\
                                               & \impl \neg \cientificaAdj \\
%
%
\intertext{\cgrule{REMOVE adj IF (NOT -1* n)} Scanning. There must be no nouns before the target.}
%
\neg (\sobreN \vee \aproximacionN) & \impl  \neg{}\masAdj \notag \\
\neg (\sobreN \vee \aproximacionN) & \impl  \neg{}\cientificaAdj \\
%
\intertext{\cgrule{REMOVE adj IF (NOT -1* n BARRIER det)} Scanning up to a barrier. There must be no nouns before the target, up to a determiner.}
%
\neg ( (\sobreN \wedge \neg \unaDet) \vee \aproximacionN) & \impl \neg{}\masAdj \notag \\
\neg ( (\sobreN \wedge \neg \unaDet) \vee \aproximacionN) & \impl \neg{}\cientificaAdj \\
%
%
\intertext{\cgrule{REMOVE adj IF (NOT -1* n CBARRIER det)} Scanning up to a careful barrier. There must be no nouns before the target, up to an unambiguous determiner.}
%
\neg ((\sobreN \wedge \neg \sobrePr \wedge \unaAny) \vee \aproximacionN) & \impl \neg{}\masAdj \notag \\ 
\neg ((\sobreN \wedge \neg \sobrePr \wedge \unaAny) \vee \aproximacionN) & \impl \neg{}\cientificaAdj 
%
%
%
\end{align}

\subsubsection{Non-matching inverted conditions}

Here we demonstrate a number of inverted rules, in which the contextual test does not match the example sentence. As a result, the action is performed unconditionally.

\cgrule{REMOVE adj IF (NOT 1 punct)} Single inverted condition, not present in initial analysis.

\cgrule{REMOVE adj IF (NOT 10 adj)} Single inverted condition, out of scope.

\cgrule{REMOVE adj IF (NOT 1 n) (NOT 10 n)} Conjunction of inverted conditions, one out of scope.

\cgrule{REMOVE adj IF (NEGATE -3 pr LINK 1 punct)} Inversion of a conjunction of conditions, some not present in initial analysis.

\begin{equation}
 \impl \neg{}\masAdj 
\end{equation}






% \begin{figure}[h]
%   \begin{itemize}
%    \item[] \begin{verbatim}REMOVE adv IF (-1 n) ;          REMOVE adj IF (-1 n) ;
% REMOVE adj IF (-1 n) ;          REMOVE adv IF (-1 n) ;\end{verbatim}
%    \end{itemize}
%    \caption{Two rules that target the same reading, in different orders.}
%    \label{fig:orderMatters}
% \end{figure}




% Also, the SAT-solver requires less rules, and these rules are simpler. 
% The ordering of rules is something we found was incompatible with a logic-based approach. We compensate this by maximising rule applications. Whether this is an advantage or disadvantage remains to be seen. Our initial experimental results are promising.

